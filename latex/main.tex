\documentclass[a4paper,10pt]{scrreprt}
\usepackage[utf8]{inputenc}
\usepackage{a4wide}

% Title Page
\title{Autonomous driving in urban environments, roundabouts}
\author{Julian-B. Scholle}


\begin{document}
\maketitle

\chapter{Introduction}
Führt in das Thema ein und berücksichtigt dabei Fragen wie:
\begin{itemize}
 \item Motivation für die Bearbeitung des Themas
 \item Prinzipielle Methoden, die zur Lösung der Aufgabenstellung verwendet werden sollen
\end{itemize}
Gibt einen Überblick über den Ausgangspunkt der Arbeit, d. h. den aktuellen Stand der Forschung im Themenbereich.\\\\
Je nach Umfang wird der Stand der Forschung oft aber auch in einem eigenen Kapitel behandelt, das meist im Anschluss an die Einleitung folgt.\\\\
Kann in mehrere Unterkapitel untergliedert werden\\\\
Die Einleitung endet mit einem Überblick über die einzelnen Kapitel der Arbeit (in der Regel 1-3 Sätze je Kapitel)

\section{Motivation}
\section{Method}
\chapter{Research}
Klare, logische Gliederung\\\\
Möglichst ausgeglichene Kapitel (bezüglich Umfang und Zahl der Unterkapitel)\\\\
Gesamte Arbeit enthält so wenig Redundanz wie möglich\\\\
Ist auch innerhalb der einzelner Kapitel oder Abschnitte sinnvoll strukturiert\\\\
Kapitel und Unterkapitel beginnen stets mit einer ganz kurzen Einleitung (in der Regel 1-3 Sätze, die erklären, was im Folgenden zu erwarten ist)\\\\
Kurze, aussagekräftige Überschriften in einheitlichem Stil\\\\
Beschreibung von Konzepten. Technische Details, wie z.B. Quellcode, umfangreiche Auflistungen, ergänzende Abbildungen usw. kommen in den Anhang.\\\\
\chapter{Evaluation}
\chapter{Conclusions}
Kann in mehrere Unterkapitel gegliedert werden\\\\
Greift Thesen oder Fragestellungen aus der Einleitung wieder auf\\\\
Fasst die Arbeit knapp und prägnant zusammen\\\\
Ordnet die Ergebnisse in Gesamtzusammenhänge ein\\\\
Zieht Schlussfolgerungen aus den erarbeiteten Ergebnissen\\\\
Kann auch eigene Bewertungen oder Meinungen enthalten\\\\
Gibt eine Ausblick auf mögliche Konsequenzen oder notwendige weitere zu lösende Probleme
\chapter{Future Work}

\end{document}          
