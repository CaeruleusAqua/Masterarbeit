% \documentclass[12pt,paper=a4,oneside,
% %headings=small,
% open=right,
% numbers=noenddot,
% titlepage=true,
% abstract=false,
% bibliography=nottotoc,	
% listof=flat,
% caption=tableheadings,	
% parskip
% ]{scrreprt}
% \usepackage[utf8]{inputenc}
% \usepackage[T1]{fontenc}  
% \usepackage{a4wide}

% \usepackage[hidelinks]{hyperref}
% \usepackage{wdok-title}

% Time-stamp: <2008-03-17T14:24:07 mxp>
\documentclass[11pt,oneside,openright]{mpreport}
\usepackage[type=master]{wdok-title}
\usepackage[utf8]{inputenc}
\usepackage[T1]{fontenc}
\usepackage{textcomp}
\usepackage{mathptmx}
\usepackage[scaled=0.9]{helvet}
\usepackage{courier}
\usepackage[hidelinks]{hyperref}
\usepackage{cleveref}
\usepackage{siunitx}
\usepackage{todonotes}
\usepackage[style=alphabetic]{biblatex}
\addbibresource{master.bib}


% Examples for the definition of convenience commands
\newcommand{\package}[1]{\texttt{#1}}
\newcommand{\foreign}[1]{\emph{#1}}
\newcommand{\q}[1]{»#1«}


\newtheorem{defi}{Definition}[chapter]
\newtheorem{satz}{Satz}[chapter]
\newtheorem{bsp}{Beispiel}[chapter]

% Title Page
\title{Autonomous driving in urban environments, roundabouts}
\author{Julian-B. Scholle}

%\supervisor{Themensteller\\
%  Betreuer}


\begin{document}
\maketitle

\tableofcontents

\chapter*{Eidesstattliche Erklärung}
Ich erkläre hiermit an Eides statt, dass ich die vorliegende Arbeit selbständig und
ohne unerlaubte fremde Hilfe angefertigt, andere als die angegebenen Quellen und
Hilfsmittel nicht benutzt habe. Die aus fremden Quellen direkt oder indirekt
übernommenen Stellen sind als solche kenntlich gemacht.
Die Arbeit wurde bisher in gleicher oder ähnlicher Form keinem anderen
Prüfungsamt vorgelegt und auch nicht veröffentlicht.

\noindent Göteborg, den \today
\begin{flushright}
$\overline{~~~~~~~~~\mbox{Julian-B. Scholle}~~~~~~~~~}$
\end{flushright}

\chapter{Introduction}
% Das autonome Fahren und die Vernetzung von Fahrzeugen mit Ihrer Umwelt sind zusammen mit der Elektromobilität die meistdiskutierten Themen der Automobilbranche.
% Zu Recht: Autonomes Fahren besitzt das Potenzial, im Mobilitätsmarkt völlig neue Strukturen entstehen zu lassen.
``Autonomous driving and the networking of vehicles with their environment are, together with electromobility, the most frequently discussed topics in the automotive sector.
Rightly so: Autonomous driving has the potential to create completely new structures in the mobility market.''
\footnote{\url{https://www2.deloitte.com/de/de/pages/consumer-industrial-products/articles/autonomes-fahren-in-deutschland.html} (03/09/2017)}

% So ebenfalls die Technische Hochschule Chalmers welche ergänzend zu Volvos “DriveMe” Projekt das Projekt
% “CampusShuttle” initiiert hat, “CampusShuttle” ist ein interdisziplinäres Forschungsprojekt der Technischen Hochschule Chalmers und der Universität Göteborg.
% Das Projekt ist dabei im ReVeRe (Chalmers Research Vehicle Resource) angesiedelt. Die Vision ist dabei ein selbstfahrendes Auto zwischen den beiden Campus der Technische Hochschule Chalmers.
So also, the Chalmers University of Technology, which has also initiated the project "CampusShuttle" in addition to Volvos' "DriveMe" project, 
is an interdisciplinary and cooperative research project at the Chalmers University of Technology and the University of Gothenburg.
The project is located in the ReVeRe (Chalmers Research Vehicle Resource). The vision is a self-driving car between the two campuses of Chalmers.

% Dabei soll, im Rahmen des Projekts, das Fahrzeug in verschiedenen Verkehrsszenarien untersucht werden. Der Fokus liegt dabei besonders auf den Stadtverkehr, das Fahrzeug muss dabei nicht
% nur in der Lage sein mit anderen Autos zu interagieren, sondern ebenfalls mit Straßenbahnen, Bussen, Fahrrädern aund allen Anderen Verkehrsteilnehmern sicher agieren. 
Within the scope of the project, the vehicle is to be examined in various traffic scenarios. The focus is on urban transport and the vehicle must not only be able to interact with other cars, 
but also be safe with trams, buses, bicycles and all other traffic users.

\section{Initial Situation}


\subsection{Test Platform}

\begin{figure}[!ht]
%\begin{center}
\caption{Test Platform Snowfox}
\includegraphics[width=\columnwidth]{bilder/snowfox.jpg}
\label{snowfox}
%\end{center}
\end{figure}

% Die in dieser Arbeit genutze Testplatform ist ein Volvo XC90 (2015) SUV, gennate Snowfox (siehe \cref{snowfox}). Diese Testpaltform ist mit vielen Sensoren zur Umfeldwarnehmung ausgestattet.
% Dazu zählen fünf Radar Sensoren, rund um das Fahrzeug. Wobei das Front Radar über eine Größere Reichweite verfügt. Sowie eine
% Stereo Kamera und ein Velodyne VLP-16 LiDAR. Die Anordnung der Sensoren kann \cref{platform} entnommen werden.

The test platform used in this work is a Volvo XC90 (2015) SUV, named Snowfox (see \cref{snowfox}). 
This test platform is equipped with many sensors for environmental monitoring. This includes five radar sensors, all around the vehicle, where front radar has a wider range.
As well as a stereo camera and a Velodyne VLP-16 LiDAR. The arrangement of the sensors can be taken from \cref{platform}.


\begin{figure}[!ht]
%\begin{center}
\caption{Snowfox Sensors}
\includegraphics[width=\columnwidth]{sensors.pdf}
\label{platform}
%\end{center}
\end{figure}

% Zusätlich zur Umfeldsensorik und der serienmäßgigen Fahrzeugsensorik (z.B. Odometer, Interialsensorik) ist im Fahrzeug ein Applanix POS LV verbaut.
% Zu Zeitpunkt des Verfassens dieser Arbeit war es leider noch nicht möglich auf die Radarsensoren und die Stereokamera zuzugreifen. 
% Daher werden im folgenden lediglich der Velodyne LiDAR und das Applanix System genauer beschrieben.
A Applanix POS LV is installed in the vehicle in addition to the environmental sensor system and the standard vehicle sensors (for example odometer, intertial sensors).
At the time of writing this work, it was unfortunately not yet possible to access the radar sensors and the stereo camera.
Therefore, only the Velodyne LiDAR and the Applanix system will be described in detail.

\subsubsection{Velodyne VLP-16 LiDAR}
% Der Velodyne VLP-16 ist ein 360 Grad 3D Laserscannermit einer Rotationsgeschwindigkeit von 5 bis 20 Umdrehungen pro Sekunde \cite{manVEL}. Er bietet ein vertikales FOV von 30 Grad, bei 2 Grad Auflösung.
% Mit einer Reichweite von 100m kann er einen Umkreis von 200m Durchmesser abdecken. Weiterhin kann der VLP-16 mit dem Applanix POS LV syncronisiert werden, was eine jitterarme Zeimessung ermöglicht.
% Eine weiter Funktion des Velodyne Sensors, ist das er auf verschiedene Messimpulse reagieren kann. Durch die Auswertung des letzten Impulses statt des Stärksten Impulses ist es Möglich durch Transparent Objekte zu sehen.
% Das ermöglicht uns im späteren Verlauf die Breite des Fahrzeues zu ermitteln, da der Velodyne durch die Glasfenster des Fahrzeues blicken kann.
% Bei einer eingestellten Geschwindigkeit von 10Hz liefert der VLP-16 eine Auflösung von 0.2 Grad bei einer Abreichung von +-3cm. Der VLP-16 ist mittig auf dem Dach des XC90 moniert, um eine möglichst hohe Positionierung
% zu erreichen, die eine Rundumsicht umd Das Fahrzeug zu erreichen. Zu beachten ist, das diese Ausrichtung für den Sensor denkbar ungünstig ist, da der Sensor ein vertikales
% Sichtfeld von -15 bis +15 Grad hat. Dadurch sind nachezu alle messungen über Null grad quasi nutzlos. Der blick auf die Herstellerseite
% \footnote{\url{http://velodyneLiDAR.com/vlp-16.html} (03/09/2017)}
% verrät, das Der VLP-16 aunteranderem auf die verwendung mit Drohnen hin konstuiert wurde, während der Größere HDL64E
% \footnote{\url{http://velodyneLiDAR.com/hdl-64e.html} (03/09/2017)}
% explizit für den Urbanen Automotivebereich beworben wird, und über ein Sichtfeld von +2 bis -24.9 Grad verfügt und somit für die Verwendung im
% Automotive bereich geeigneter erscheint. Die dabei entstehenden Probleme werden später diskutiert.

The Velodyne VLP-16 is a 360 degree 3D laser scanner with a rotational speed of 5 to 20 revolutions per second \cite{manVEL}.
It provides a vertical FOV of 30 degrees, at 2 degrees resolution.
With a range of 100m it can cover a circumference of 200m diameter. 
Furthermore, the VLP-16 can be syncronized with the Applanix POS LV, which allows a correct heading of the data. 
A further function of the Velodyne sensor is that it can react to different measuring pulses. 
By evaluating the last pulse instead of the strongest pulse it is possible to see through transparent objects. 
This allows to determine the width of the vehicle in a later part of this work, as the Velodyne can look through the glass windows of the vehicle. 
At a set speed of 10Hz, the VLP-16 provides a resolution of 0.2 degrees with a variance of +- 3cm. 
The VLP-16 is centered on the roof of the XC90 in order to achieve the highest possible positioning to achieve a panoramic view of the vehicle.
It is important to note that this alignment is unacceptable to the sensor because the sensor has a vertical field of view of -15 to +15 degrees.
As a result, all measurements over zero degrees are practically useless. The view of the manufacturer side
\footnote{\url{http://velodyneLiDAR.com/vlp-16.html} (03/09/2017)}
reveals that the VLP-16 has been constituted for use with drones, while the larger HDL64E
\footnote{\url{http://velodyneLiDAR.com/hdl-64e.html} (03/09/2017)}
is advertised explicitly for the urban automotive sector, and has a field of view of +2 to -24.9 degrees and thus for use in the
Automotive sector appears to be more appropriate. The resulting problems will be discussed later.


\subsubsection{Applanix POS LV}
% Das POS LV ist ein kompaktes Positions- und Orientierungssystem. Es Offeriert stabile, zuverlässige und reproduzierbare Positionierungslösungen für landgestützte Fahrzeuganwendungen.
% Das POS LV liefert dabei eine Inertialsensork und Odometrie gestützte Positionsmessung mit einer Genauigkeit von bis zu 0.3m (bis zu 0.035m bei verwendung von der der RTK - Korrektur).
% Im weiteren Verlauf wird außerdem das vom POS LV gelieferte Heading genutzt, welches eine Genauigkeit von 0.2 Grad liefert. Auch nach ausfall des GPS-Signals kann das POS-LV durch sein
% Odeomerter und der Inertialsensork eine Position liefern. Diese wird jedoch über die Zeit schlechter, so das 60Sek nach Ausfall des GPS-Signals nich eine Genauigkeit von 2.51m erwartet
% werden kann.\cite{manAP}
The POS LV is a compact position and orientation system. It offers stable, reliable and reproducible positioning solutions for land-based vehicle applications. The POS LV provides an 
inertial sensor and odometry based position measurement with an accuracy of up to 0.3m (up to 0.035m when using the RTK correction).
Furthermore, the heading delivered by the POS LV is also used, which provides an accuracy of 0.2 degrees. Even after losing the GPS signal,
the POS-LV can provide a position through its odometer and the inertial sensor. However, this will deteriorate over time so that an accuracy of 2.51m
can be expected 60 seconds after the GPS signal is lost. \cite{manAP}


\section{Research Goal}
% Da das Autonome Fahren ein sehr weites, indisziplinäres thema ist, ist es Offensichtlich. das nicht alles in dieser Arbreit abgehandelt werden kann.
% Im Rahmen der darpah Chalenge wurden beiteits viele Veröffentlichungen zu diesem Thema erstellet.
% Was im rahmen dieser Veröffenticghungten noch nicht berhandelt wurde, sit dei Handhabung von Kreisverkehre, mit atonomen Fahrzeugen.
% Ziel dieser Arbeit ist es Daher zu analysieren, welche Sensorausstattung für die beobachtung von Kreisverkehren vonnöten ist, bzw. ob die vorhandenne Sensorausstattung des ReVeRe Testfahrzeuges Snowfox
% als ausreichend betrachtet werden kann.
Since autonomous driving is a very wide, indisciplinary subject, it is obvious, that not everything can be dealt within this work.
Within the scope of \acs{DARPA} Chalenge many papers were published on this subject. What has not yet been explicitly discussed in these publications
is the explicit handling of roundabouts, with autonomic vehicles, also the author is not aware of any further publications. 
So the aim of this thesis is therefore to analyze what sensor equipment is necessary for the observation of roundabouts,
or whether the existing sensor equipment of the ReVeRe test vehicle Snowfox can be regarded as sufficient.


\chapter{Related Work}

\section{Autonomous Driving}

\section{Roundabouts in Law}
% In Deutschland gibt es kein Gesetzt was den genauen Konstuktion von Kreisverkehren vorschreibt.
% Stattdessen werden die Elemente der Landstraßen und Stadtstraßen in Richtlinien für die Anlage von Landstraßen (RAL) bzw.
% den Richtlinien für die Anlage von Stadtstraßen (RASt) behandelt. Diese Richtlinien sind auch für die
% Wahl einer zweckmäßigen Knotenpunktart bei der Verknüpfung von Straßen maßgebend. Für diese Abreit sind die 
% Richtlinien für die Anlage von Stadtstraßen (RASt) relevant. Die dort behandelten Abwägungsüberlegungen orientieren sich an verkehrlichen Größen, umfeldbezogenen Merkmalen,
% wirtschaftlichen Kriterien und raumordnerischen oder städtebaulichen Vorgaben. Die Richtlinien regeln auch grundlegend die entwurfstechnische und betriebliche Ausbildung von Kreisverkehren.
%
In Germany, there is no law stipulating the exact construction of roundabouts.
Instead, the elements of the rural roads and city streets are dealt with in Directives for the Design of rural roads \cite{ral13}
and the Directives for the Design of Urban Roads \cite{rast06}. These guidelines are also relevant to the choice of a convenient junction type when linking roads.
The considerations discussed there are based on traffic variables, area-related characteristics, economic criteria and spatial planning or urban planning requirements. 
The guidelines also regulate the basic design and operational formation of roundabouts.
The  Directives for the Design of Urban Roads \cite{rast06} are relevant for this dispute. Since the access the RASt ist limited, most of the information is coming from
\cite{man06} whereupon RASt is based on.
\subsection{Elements of a Roundabout}

\begin{figure}[!ht]
%%\begin{center}
\caption{Definition of individual design elements and dimensions of a roundabout \cite{man06}}
\includegraphics[width=0.7\textwidth]{bilder/kreisverkehr.png} %70% der Textbreite
\label{roundabout_parts}
%%\end{center}
\end{figure}


\begin{defi}[roundabout island]
The roundabout island is the constructional area in the middle of the roundabout, which is surrounded by vehicles.
For miniature roundabouts, the roundabout island is crossable. \cite{man06}
\end{defi}

\begin{defi}[circular path]
%Die Kreisfahrbahn ist die Fahrbahn, die zum Umfahren der Kreisinsel
%dient. Ein ggf. vorhandener Innenring ist verkehrsrechtlich nicht Be-
%standteil der Kreisfahrbahn (VwV-StVO zu §9a V., Rn. 5).

%%TODO
The circular path is the road that serves to drive the roundabout island. An inner ring, if present, is not part of the circular path (VwV-StVO zu §9a V., Rn. 5). \cite{man06}
\end{defi}

\begin{defi}[circular ring with ($B_K$)]
% Die bauliche Breite umfasst die Kreisfahrbahn und einen ggf. gepflasterten
% Innenring. Sie ist abhängig vom Außendurchmesser und der angestrebten
% Verkehrsführung (ein- oder zweistreifig). Die Randstreifenbreite orientiert
% sich an der maßgebenden durchgehenden Fahrbahn.
The structural width includes the circular track and a paved inner ring, if any. It is dependent on the outer diameter and the desired traffic routeing (one or two lanes). 
The edge strip width is oriented on the relevant continuous roadway. \cite{man06}
\end{defi}

\begin{defi}[outer diameter ($D$)]
The outer diameter is measured at the outer edge of the circular ring. It is the essential measure for describing the size of the roundabout. \cite{man06}
\end{defi}

\begin{defi}[inner diameter ($D_I$)]
The inner diameter is the diameter of the roundabout island. \cite{man06}
\end{defi}

\begin{defi}[road divider]
% Der Fahrbahnteiler ist die baulich ausgeführte Insel zwischen Kreisausfahrt
% und -zufahrt einer angeschlossenen Straße. Er dient der Trennung der
% Kreisaus- und -zufahrten, der Führung des Verkehrs sowie den Fußgängern
% und Radfahrern als Überquerungshilfe.
%
The road divider is the structurally designed island between the circular exit
and circular driveway. It serves to separate the circular exit and circular driveway, the management of the traffic, as well as the pedestrians and cyclists as cross-bordering aid. \cite{man06}
\end{defi}

\begin{defi}[lane width of the circular driveway ($B_Z$) and circular exit ($B_A$)]
% Die Breite der Kreiszufahrt und Ausfahrt wird am Beginn der Eckausrundung gemessen.
The width of the circular driveway and exit is measured at the beginning of the corner. \cite{man06}
\end{defi}

\begin{defi}[Corner rounding radius ($R_Z$ and $R_A$) ]
% Dies ist der Radius der Ausrundung am rechten Fahrbahnrand zwischen 
% der Kreiszufahrt und der Kreisfahrbahn. Bei einem Korbbogen mit einer
% Radienfolge aus drei unterschiedlichen Radien ist RZ der Radius R2 des
% mittleren Bogens. Bei der Ausbildung des Fahrbahnrandes als Schleppkurve ist RZ der kleinste Radius des Fahrbahnrandes.
% 
This is the radius of the rounding at the right edge of the road between the circular driveway and the circular path.
For a elliptical arch with a radius sequence of three different radii, $R_Z$ is the radius $R_2$ of the central arc.
When the road edge is formed as a tractrix, $R_Z$ is the smallest radius of the road edge. \cite{man06}
\end{defi}

\subsection{Types of Roundabouts}
There are several types of roundabouts, which are differentiated by the different application criteria and the partly different design principles according to the situation inside and outside built areas.
Furthermore, a division is made as a function of its size. \cite{man06}


\subsubsection{Mini Roundabout}

\begin{figure}[!ht]
%\begin{center}
\caption{Mini Roundabout \cite{man06}}
\includegraphics[width=0.5\textwidth]{bilder/mini_roundabout.png} %70% der Textbreite
\label{roundabout_mini}
%\end{center}
\end{figure}

Within built-up areas, smaller outer diameters are possible under certain conditions.
These roundabouts are called mini roundabout. The roundabout island must then be capable of being passed over.
The outer diameter should be at least 13 m, so that the circular island does not become too small.
Larger outer diameters make driving easier. Outer diameters of more than 22m, however, do not offer any transport advantages.
From an outside diameter of about 22 m, therefore, the installation of a small roundabout with 26 m is generally more convenient.
Bypasses are generally not required in the areas where mini roundabout can be used.


\subsubsection{Small Roundabout}

\begin{figure}[!ht]
%\begin{center}
\caption{Small Roundabout \cite{man06}}
\includegraphics[width=0.5\textwidth]{bilder/small_roundabout.png} %70% der Textbreite
\label{roundabout_small}
%\end{center}
\end{figure}

The small roundabout has a single lane circular path and single lane circular driveways and exits. The roundabout island is not passable.
The outer diameter must be at least 26 m. Bypasses can be set up for driving geometric reasons or to increase performance.


\subsubsection{Two-lane Passable Roundabout}

\begin{figure}[!ht]
%\begin{center}
\caption{Two-lane Passable Roundabout \cite{man06}}
\includegraphics[width=0.5\textwidth]{bilder/twolaned_roundabout.png} %70% der Textbreite
\label{roundabout_twolaned}
%\end{center}
\end{figure}

% Reicht die Kapazität des Kleinen Kreisverkehrs nicht aus und kann diese nicht durch die Anlage von Bypässen sicher gestellt werden,
% kann die Kreisfahrbahn eines Kleinen Kreisverkehrs zweistreifig befahrbar ausgebildet werden.
% An einem solchen Kreisverkehr ist die Kreisfahrbahn so breit, dass Pkw im Kreis nebeneinander fahren können.
% Wird eine weitere Erhöhung der Kapazität erforderlich, können einzelne Kreiszufahrten ebenfalls zweistreifig ausgeführt werden,
% wenn Fußgänger und Radfahrer regelmäßig nicht zu berücksichtigen sind. Kreisausfahrten werden aus Sicherheitsgründen immer einstreifig ausgeführt.
% Aus geometrischen Gründen muss der Außendurchmesser bei zweistreifiger Befahrbarkeit mindestens 40 m betragen.
%
If the capacity of the small roundabout is not sufficient and can not be ensured by the installation of bypasses,
the circular path of a small roundabout can be designed to be two-lane driveable.
At such a roundabout, the circular path is so wide that cars can travel side by side in a circle.
If a further increase in the capacity is required, individual circular driveway can also be carried out in two lanes, if pedestrians and cyclists are not to be considered regularly.
For safety reasons, circular exits are always carried out in single lanes.
For geometrical reasons, the outer diameter must be at least 40 m for two-laned accessibility.


\subsubsection{Large Roundabout}


\begin{figure}[!ht]
%\begin{center}
\caption{Large Roundabout \cite{man06}}
\includegraphics[width=0.5\textwidth]{bilder/large_roundabout.png} %70% der Textbreite
\label{roundabout_large}
%\end{center}
\end{figure}

%Große Kreisverkehre mit zwei oder mehreren durch Markierungen gekennzeichnete Fahrstreifen auf der Kreisfahrbahn sollen bei enger
%Abstimmung zwischen Knotenpunktentwurf und Verkehrssteuerung nur mit Lichtsignalanlage betrieben wer den.
Large Roundabouts with two or more lanes marked by markers on the circular path should be operated with a light signaling system only,
if the nodal point design and traffic control are closely coordinated.


\section{Middleware OpenDAVINCI}
Autonome Software ist typischer weise ein verteiletes System, auf heutigen Fahrzeugen basiert dieses System auf ECUs und Bussystemen wie CAN,LIN.
Verteilete Sysoftware vereinfacht es komplexe Komponenten innheralb des Systems zu integreieren. Im bereich des Autonomen Fahrens iust der Historische aufbau von Fahrzeugen
mit ECU'S und can jedoch nicht optimal. Um die vielen benötigekten Komponenten zu handhaben, ist es von vorteil Komponenten auch innerhalb eines ECU's bzw einer Recheneinheit
zu entkoppeln. Für diesen Zweck gibt es beireits mehrer middelwares die unteranderm die Kommunikation innerhalb der Komponenten handhaben und abstrahieren.
m Rahmen des Copplar Projekets, wird hier die OpenDaVINCI middleware genutzt. OpenDaVINCI ist eine echtzeitfähige laufzeitumgebeung konzipiert für Autonome Fahrzeuge.
OpenDaVINCI basiert auf Hesperia \cite{Berger2010}. Die Kommunikation zwischen den Komponenten basiert in OpenDaVINCI UDP Multicast, welches eine Echtzeitfähige Kommunikation zwischen
den Komponenten Ermöglicht \cite{Kurose2013}. Für die Kommunikation bietet OpenDaVINCI Time-triggered sender und Data-triggered receiver an, von welchem in folgenden der Data-triggered receiver
für die Anbindung der Software genutzt wird. Weiterhin bietet OpenDaVINCI viele weiter Funktionalitäten die das Handling von World Geodetic System 1984 (WGS84) Korridinaten an, welches für die Umwandlung
von GPS koordinaten in lokale kartesische genutzt werden kann. Dazu ist die Angabe einer referenz GPS postion nötig, welche um den Berechnugsfehler klein zu halten,
nicht zu weit entfernt sein sollte.


\section{Test Platform}
Die in dieser Abreit genutze Testplatform ist ein Volvo XC90 (2015) SUV. Dies Testpaltform ist mit vielen Sensoren zur Umfeldwarnehmung ausgestattet.
Dazu zählen fünf Radar Sensoren, rund um das Fahrzeug. Wobei das Front Radar über eine Größere Reichweite verfügt. Sowie eine
Stereo Kamera und ein Velodyne VLP-16 LiDAR. Die Anordnung der Sensoren kann \cref{platform} entnommen werden.


\begin{figure}[!ht]
%\begin{center}
\caption{Test Platform}
\includegraphics[width=\columnwidth]{sensors.pdf}
\label{platform}
%\end{center}
\end{figure}


Zusätlich zur Fahrzeuginternen Sensorik (Odometer, Interialsensorik) ist im Fahrzeug ein Applanix POS LV verbaut. Zu Zeitpunkt des Verfassens dieser Arbeit war es leider nicht möglich auf
die Radarsensoren und die Stereokamera zuzugreifen. Daher werden im folgenden lediglich der Velodyne Lidar und das Applanix System genauer beschrieben.

\subsection{Velodyne VLP-16 LiDAR}
Der Velodyne VLP-16 ist ein 360 Grad 3D Laserscannermit einer Rotationsgeschwindigkeit von 5 bis 20 Umdrehungen pro Sekunde. Er bietet ein vertikales FOV von 30 Grad, bei 2 Grad Auflösung.
Mit einer Reichweite von 100m kann er einen Umkreis von 200m Durchmesser abdecken. Weiterhin kann der VLP-16 mit dem Applanix POS LV syncronisiert werden, was eine jitterarme Zeimessung ermöglicht.
Eine weiter Funktion des Velodyne Sensors, ist das er auf verschiedene Messimpulse reagieren kann. Durch die Auswertung des letzten Impulses statt des Stärksten Impulses ist es Möglich durch Transparent Objekte zu sehen.
Das ermöglicht uns im späteren Verlauf die Breite des Fahrzeues zu ermitteln, da der Velodyne durch die Glasfenster des Fahrzeues blicken kann.
Bei einer eingestellten Geschwindigkeit von 10Hz liefert der VLP-16 eine Auflösung von 0.2 Grad bei einer Abreichung von +-3cm. Der VLP-16 ist mittig auf dem Dach des XC90 moniert, um eine möglichst hohe Positionierung
zu erreichen, die eine Rundumsicht umd Das Fahrzeug zu erreichen. Zu beachten ist, das diese Ausrichtung für den Sensor denkbar ungünstig ist, da der Sensor ein vertikales
Sichtfeld von -15 bis +15 Grad hat. Dadurch sind nachezu alle messungen über Null grad quasi nutzlos. Der blick auf die Herstellerseite
\footnote{\url{http://velodynelidar.com/vlp-16.html} (03/09/2017)}
verrät, das Der VLP-16 aunteranderem auf die verwendung mit Drohnen hin konstuiert wurde, während der Größere HDL64E
\footnote{\url{http://velodynelidar.com/hdl-64e.html} (03/09/2017)}
explizit für den Urbanen Automotivebereich beworben wird, und über ein Sichtfeld von +2 bis -24.9 Grad verfügt. Die dabei entstehenden Probleme werden später diskutiert.



\subsection{Applanix POS LV}
Das POS LV ist ein kompaktes Positions- und Orientierungssystem. Es Offeriert stabile, zuverlässige und reproduzierbare Positionierungslösungen für landgestützte Fahrzeuganwendungen.
Das POS LV liefert dabei eine Inertialsensork und Odometrie gestützte Positionsmessung mit einer Genauigkeit von bis zu 0.3m (bis zu 0.035m bei verwendung von der der RTK - Korrektur).
Im weiteren Verlauf wird außerdem das vom POS LV gelieferte Heading genutzt, welches eine Genauigkeit von 0.2 Grad liefert. Auch nach ausfall des GPS-Signals kann das POS-LV durch sein
Odeomerter und der Inertialsensork eine Position liefern. Diese wird jedoch über die Zeit schlechter, so das 60Sek nach Ausfall des GPS-Signals nich eine Genauigkeit von 2.51m erwartet
werden kann.\cite{manAP}








\chapter{Methodology}
Gefragt wird hier nach den Kriterien dafür, welche Methode für eine bestimmte Art der Anwendung geeignet ist, warum eine bestimmte Methode angewandt werden muss oder angewendet wird und keine andere. Verständnisfragen zum methodischen Weg werden hier geklärt. Die Methodologie ist demnach eine Metawissenschaft und somit eine Teildisziplin der Wissenschaftstheorie. Demgegenüber bezeichnet Methodik das Methodenwissen des Praktikers oder des Wissenschaftlers.




\chapter{Research}
Klare, logische Gliederung\\\\
Möglichst ausgeglichene Kapitel (bezüglich Umfang und Zahl der Unterkapitel)\\\\
Gesamte Arbeit enthält so wenig Redundanz wie möglich\\\\
Ist auch innerhalb der einzelner Kapitel oder Abschnitte sinnvoll strukturiert\\\\
Kapitel und Unterkapitel beginnen stets mit einer ganz kurzen Einleitung (in der Regel 1-3 Sätze, die erklären, was im Folgenden zu erwarten ist)\\\\
Kurze, aussagekräftige Überschriften in einheitlichem Stil\\\\
Beschreibung von Konzepten. Technische Details, wie z.B. Quellcode, umfangreiche Auflistungen, ergänzende Abbildungen usw. kommen in den Anhang.\\\\
\chapter{Evaluation}
\chapter{Conclusions}
Kann in mehrere Unterkapitel gegliedert werden\\\\
Greift Thesen oder Fragestellungen aus der Einleitung wieder auf\\\\
Fasst die Arbeit knapp und prägnant zusammen\\\\
Ordnet die Ergebnisse in Gesamtzusammenhänge ein\\\\
Zieht Schlussfolgerungen aus den erarbeiteten Ergebnissen\\\\
Kann auch eigene Bewertungen oder Meinungen enthalten\\\\
Gibt eine Ausblick auf mögliche Konsequenzen oder notwendige weitere zu lösende Probleme
\chapter{Future Work}

\printbibliography

\end{document}          
