\PassOptionsToPackage{hyphens}{url}
\PassOptionsToPackage{pdftex,dvipsnames}{xcolor}
\documentclass[11pt,oneside,openright]{mpreport}
\usepackage[type=master]{wdok-title}
\usepackage[utf8]{inputenc}
\usepackage[T1]{fontenc}
\usepackage[ngerman,english]{babel}
\usepackage{textcomp}
\usepackage{mathptmx}
\usepackage[scaled=0.9]{helvet}
\usepackage{courier}
\usepackage[hidelinks]{hyperref}
\usepackage{cleveref}
\usepackage{siunitx}
\usepackage{listings}
\usepackage[bibencoding=auto,backend=biber,babel=other]{biblatex}
\usepackage{csquotes}

%\usepackage[style=numeric]{biblatex}
\addbibresource{main.bib}
\usepackage{mathtools}
\usepackage{svg}
\usepackage{tikz}
\usepackage{tikz-uml}
\usepackage{pgfplots}
\pgfplotsset{compat=1.14}
\usepackage{xargs}                      % Use more than one optional parameter in a new commands
\usepackage{acronym}

\usepackage[colorinlistoftodos,prependcaption,textsize=tiny]{todonotes}
\newcommandx{\unsure}[2][1=]{\todo[linecolor=red,backgroundcolor=red!25,bordercolor=red,#1]{#2}}
\newcommandx{\change}[2][1=]{\todo[linecolor=blue,backgroundcolor=blue!25,bordercolor=blue,#1]{#2}}
\newcommandx{\info}[2][1=]{\todo[linecolor=OliveGreen,backgroundcolor=OliveGreen!25,bordercolor=OliveGreen,#1]{#2}}
\newcommandx{\improvement}[2][1=]{\todo[linecolor=Plum,backgroundcolor=Plum!25,bordercolor=Plum,#1]{#2}}
\newcommandx{\thiswillnotshow}[2][1=]{\todo[disable,#1]{#2}}

\DeclareMathOperator{\atantwo}{atan2}

% Examples for the definition of convenience commands
\newcommand{\package}[1]{\texttt{#1}}
\newcommand{\foreign}[1]{\emph{#1}}
\newcommand{\q}[1]{»#1«}


\newtheorem{defi}{Definition}[chapter]
\newtheorem{satz}{Satz}[chapter]
\newtheorem{bsp}{Beispiel}[chapter]

% Title Page
\title{Autonomous Driving in Urban Centers - Roundabout Monitoring}
\author{Julian-B. Scholle}

%\supervisor{Themensteller\\
%  Betreuer}

\pagenumbering{Roman}


\begin{document}

\maketitle



\chapter*{Eidesstattliche Erklärung}
Ich erkläre hiermit an Eides statt, dass ich die vorliegende Arbeit selbständig und
ohne unerlaubte fremde Hilfe angefertigt, andere als die angegebenen Quellen und
Hilfsmittel nicht benutzt habe. Die aus fremden Quellen direkt oder indirekt
übernommenen Stellen sind als solche kenntlich gemacht.
Die Arbeit wurde bisher in gleicher oder ähnlicher Form keinem anderen
Prüfungsamt vorgelegt und auch nicht veröffentlicht.

\noindent Göteborg, den \today
\begin{flushright}
$\overline{~~~~~~~~~\mbox{Julian-B. Scholle}~~~~~~~~~}$
\end{flushright}


\chapter*{Acknowledgements}
Danken möchte ich außerdem besonders Associate professor Christian Berger und J. Prof Sebastian Zug, führ ihre Organisatorische und Fachliche Unterstützung während und besonders im Vorfeld dieser Abreit und
dem ``Deutschen Akademischen Austauschdienst'' - DAAD für ihre finanzielle Unterstüzung während meines Aufenthaltes in Göteborg.
Weiterhin danken möchte ich natürlich allen weiteren Kollegen aus Götborg, welche mich bei meiner Arbeit fachlich und moralisch untersützt haben.


\chapter*{Index of Abbreviations}

\begin{acronym}[Bash]
 \acro{DBSCAN} {Density-Based Spatial Clustering of Applications with Noise}
 \acro{SVD}{Singular Value Decomposition}
 \acro{LLSQ}{linear least squares}
 \acro{RANSAC}{Random Sample Consensus}
 \acro{MKS}{Multi-Body simulation}
 \acro{ACC}{Adaptive Cruise Control}
 \acro{DARPA}{Defense Advanced Research Projects Agency}
 \acro{CTRV}{Constant Turn Rate and Velocity}
 \acro{RASt}{Directives for the Design of Urban Roads}
 \end{acronym}

\tableofcontents
\clearpage 

\pagenumbering{arabic}

\chapter{Introduction}
% Das autonome Fahren und die Vernetzung von Fahrzeugen mit Ihrer Umwelt sind zusammen mit der Elektromobilität die meistdiskutierten Themen der Automobilbranche.
% Zu Recht: Autonomes Fahren besitzt das Potenzial, im Mobilitätsmarkt völlig neue Strukturen entstehen zu lassen.
``Autonomous driving and the networking of vehicles with their environment are, together with electromobility, the most frequently discussed topics in the automotive sector.
Rightly so: Autonomous driving has the potential to create completely new structures in the mobility market.''
\footnote{\url{https://www2.deloitte.com/de/de/pages/consumer-industrial-products/articles/autonomes-fahren-in-deutschland.html} (03/09/2017)}

% So ebenfalls die Technische Hochschule Chalmers welche ergänzend zu Volvos “DriveMe” Projekt das Projekt
% “CampusShuttle” initiiert hat, “CampusShuttle” ist ein interdisziplinäres Forschungsprojekt der Technischen Hochschule Chalmers und der Universität Göteborg.
% Das Projekt ist dabei im ReVeRe (Chalmers Research Vehicle Resource) angesiedelt. Die Vision ist dabei ein selbstfahrendes Auto zwischen den beiden Campus der Technische Hochschule Chalmers.
So also, the Chalmers University of Technology, which has also initiated the project "CampusShuttle" in addition to Volvos' "DriveMe" project, 
is an interdisciplinary and cooperative research project at the Chalmers University of Technology and the University of Gothenburg.
The project is located in the ReVeRe (Chalmers Research Vehicle Resource). The vision is a self-driving car between the two campuses of Chalmers.

% Dabei soll, im Rahmen des Projekts, das Fahrzeug in verschiedenen Verkehrsszenarien untersucht werden. Der Fokus liegt dabei besonders auf den Stadtverkehr, das Fahrzeug muss dabei nicht
% nur in der Lage sein mit anderen Autos zu interagieren, sondern ebenfalls mit Straßenbahnen, Bussen, Fahrrädern aund allen Anderen Verkehrsteilnehmern sicher agieren. 
Within the scope of the project, the vehicle is to be examined in various traffic scenarios. The focus is on urban transport and the vehicle must not only be able to interact with other cars, 
but also be safe with trams, buses, bicycles and all other traffic users.

\section{Initial Situation}


\subsection{Test Platform}

\begin{figure}[!ht]
%\begin{center}
\caption{Test Platform Snowfox}
\includegraphics[width=\columnwidth]{bilder/snowfox.jpg}
\label{snowfox}
%\end{center}
\end{figure}

% Die in dieser Arbeit genutze Testplatform ist ein Volvo XC90 (2015) SUV, gennate Snowfox (siehe \cref{snowfox}). Diese Testpaltform ist mit vielen Sensoren zur Umfeldwarnehmung ausgestattet.
% Dazu zählen fünf Radar Sensoren, rund um das Fahrzeug. Wobei das Front Radar über eine Größere Reichweite verfügt. Sowie eine
% Stereo Kamera und ein Velodyne VLP-16 LiDAR. Die Anordnung der Sensoren kann \cref{platform} entnommen werden.

The test platform used in this work is a Volvo XC90 (2015) SUV, named Snowfox (see \cref{snowfox}). 
This test platform is equipped with many sensors for environmental monitoring. This includes five radar sensors, all around the vehicle, where front radar has a wider range.
As well as a stereo camera and a Velodyne VLP-16 LiDAR. The arrangement of the sensors can be taken from \cref{platform}.


\begin{figure}[!ht]
%\begin{center}
\caption{Snowfox Sensors}
\includegraphics[width=\columnwidth]{sensors.pdf}
\label{platform}
%\end{center}
\end{figure}

% Zusätlich zur Umfeldsensorik und der serienmäßgigen Fahrzeugsensorik (z.B. Odometer, Interialsensorik) ist im Fahrzeug ein Applanix POS LV verbaut.
% Zu Zeitpunkt des Verfassens dieser Arbeit war es leider noch nicht möglich auf die Radarsensoren und die Stereokamera zuzugreifen. 
% Daher werden im folgenden lediglich der Velodyne LiDAR und das Applanix System genauer beschrieben.
A Applanix POS LV is installed in the vehicle in addition to the environmental sensor system and the standard vehicle sensors (for example odometer, intertial sensors).
At the time of writing this work, it was unfortunately not yet possible to access the radar sensors and the stereo camera.
Therefore, only the Velodyne LiDAR and the Applanix system will be described in detail.

\subsubsection{Velodyne VLP-16 LiDAR}
% Der Velodyne VLP-16 ist ein 360 Grad 3D Laserscannermit einer Rotationsgeschwindigkeit von 5 bis 20 Umdrehungen pro Sekunde \cite{manVEL}. Er bietet ein vertikales FOV von 30 Grad, bei 2 Grad Auflösung.
% Mit einer Reichweite von 100m kann er einen Umkreis von 200m Durchmesser abdecken. Weiterhin kann der VLP-16 mit dem Applanix POS LV syncronisiert werden, was eine jitterarme Zeimessung ermöglicht.
% Eine weiter Funktion des Velodyne Sensors, ist das er auf verschiedene Messimpulse reagieren kann. Durch die Auswertung des letzten Impulses statt des Stärksten Impulses ist es Möglich durch Transparent Objekte zu sehen.
% Das ermöglicht uns im späteren Verlauf die Breite des Fahrzeues zu ermitteln, da der Velodyne durch die Glasfenster des Fahrzeues blicken kann.
% Bei einer eingestellten Geschwindigkeit von 10Hz liefert der VLP-16 eine Auflösung von 0.2 Grad bei einer Abreichung von +-3cm. Der VLP-16 ist mittig auf dem Dach des XC90 moniert, um eine möglichst hohe Positionierung
% zu erreichen, die eine Rundumsicht umd Das Fahrzeug zu erreichen. Zu beachten ist, das diese Ausrichtung für den Sensor denkbar ungünstig ist, da der Sensor ein vertikales
% Sichtfeld von -15 bis +15 Grad hat. Dadurch sind nachezu alle messungen über Null grad quasi nutzlos. Der blick auf die Herstellerseite
% \footnote{\url{http://velodyneLiDAR.com/vlp-16.html} (03/09/2017)}
% verrät, das Der VLP-16 aunteranderem auf die verwendung mit Drohnen hin konstuiert wurde, während der Größere HDL64E
% \footnote{\url{http://velodyneLiDAR.com/hdl-64e.html} (03/09/2017)}
% explizit für den Urbanen Automotivebereich beworben wird, und über ein Sichtfeld von +2 bis -24.9 Grad verfügt und somit für die Verwendung im
% Automotive bereich geeigneter erscheint. Die dabei entstehenden Probleme werden später diskutiert.

The Velodyne VLP-16 is a 360 degree 3D laser scanner with a rotational speed of 5 to 20 revolutions per second \cite{manVEL}.
It provides a vertical FOV of 30 degrees, at 2 degrees resolution.
With a range of 100m it can cover a circumference of 200m diameter. 
Furthermore, the VLP-16 can be syncronized with the Applanix POS LV, which allows a correct heading of the data. 
A further function of the Velodyne sensor is that it can react to different measuring pulses. 
By evaluating the last pulse instead of the strongest pulse it is possible to see through transparent objects. 
This allows to determine the width of the vehicle in a later part of this work, as the Velodyne can look through the glass windows of the vehicle. 
At a set speed of 10Hz, the VLP-16 provides a resolution of 0.2 degrees with a variance of +- 3cm. 
The VLP-16 is centered on the roof of the XC90 in order to achieve the highest possible positioning to achieve a panoramic view of the vehicle.
It is important to note that this alignment is unacceptable to the sensor because the sensor has a vertical field of view of -15 to +15 degrees.
As a result, all measurements over zero degrees are practically useless. The view of the manufacturer side
\footnote{\url{http://velodyneLiDAR.com/vlp-16.html} (03/09/2017)}
reveals that the VLP-16 has been constituted for use with drones, while the larger HDL64E
\footnote{\url{http://velodyneLiDAR.com/hdl-64e.html} (03/09/2017)}
is advertised explicitly for the urban automotive sector, and has a field of view of +2 to -24.9 degrees and thus for use in the
Automotive sector appears to be more appropriate. The resulting problems will be discussed later.


\subsubsection{Applanix POS LV}
% Das POS LV ist ein kompaktes Positions- und Orientierungssystem. Es Offeriert stabile, zuverlässige und reproduzierbare Positionierungslösungen für landgestützte Fahrzeuganwendungen.
% Das POS LV liefert dabei eine Inertialsensork und Odometrie gestützte Positionsmessung mit einer Genauigkeit von bis zu 0.3m (bis zu 0.035m bei verwendung von der der RTK - Korrektur).
% Im weiteren Verlauf wird außerdem das vom POS LV gelieferte Heading genutzt, welches eine Genauigkeit von 0.2 Grad liefert. Auch nach ausfall des GPS-Signals kann das POS-LV durch sein
% Odeomerter und der Inertialsensork eine Position liefern. Diese wird jedoch über die Zeit schlechter, so das 60Sek nach Ausfall des GPS-Signals nich eine Genauigkeit von 2.51m erwartet
% werden kann.\cite{manAP}
The POS LV is a compact position and orientation system. It offers stable, reliable and reproducible positioning solutions for land-based vehicle applications. The POS LV provides an 
inertial sensor and odometry based position measurement with an accuracy of up to 0.3m (up to 0.035m when using the RTK correction).
Furthermore, the heading delivered by the POS LV is also used, which provides an accuracy of 0.2 degrees. Even after losing the GPS signal,
the POS-LV can provide a position through its odometer and the inertial sensor. However, this will deteriorate over time so that an accuracy of 2.51m
can be expected 60 seconds after the GPS signal is lost. \cite{manAP}


\section{Research Goal}
% Da das Autonome Fahren ein sehr weites, indisziplinäres thema ist, ist es Offensichtlich. das nicht alles in dieser Arbreit abgehandelt werden kann.
% Im Rahmen der darpah Chalenge wurden beiteits viele Veröffentlichungen zu diesem Thema erstellet.
% Was im rahmen dieser Veröffenticghungten noch nicht berhandelt wurde, sit dei Handhabung von Kreisverkehre, mit atonomen Fahrzeugen.
% Ziel dieser Arbeit ist es Daher zu analysieren, welche Sensorausstattung für die beobachtung von Kreisverkehren vonnöten ist, bzw. ob die vorhandenne Sensorausstattung des ReVeRe Testfahrzeuges Snowfox
% als ausreichend betrachtet werden kann.
Since autonomous driving is a very wide, indisciplinary subject, it is obvious, that not everything can be dealt within this work.
Within the scope of \acs{DARPA} Chalenge many papers were published on this subject. What has not yet been explicitly discussed in these publications
is the explicit handling of roundabouts, with autonomic vehicles, also the author is not aware of any further publications. 
So the aim of this thesis is therefore to analyze what sensor equipment is necessary for the observation of roundabouts,
or whether the existing sensor equipment of the ReVeRe test vehicle Snowfox can be regarded as sufficient.


\chapter{Related Work}

\section{Autonomous Driving}

\section{Roundabouts in Law}
% In Deutschland gibt es kein Gesetzt was den genauen Konstuktion von Kreisverkehren vorschreibt.
% Stattdessen werden die Elemente der Landstraßen und Stadtstraßen in Richtlinien für die Anlage von Landstraßen (RAL) bzw.
% den Richtlinien für die Anlage von Stadtstraßen (RASt) behandelt. Diese Richtlinien sind auch für die
% Wahl einer zweckmäßigen Knotenpunktart bei der Verknüpfung von Straßen maßgebend. Für diese Abreit sind die 
% Richtlinien für die Anlage von Stadtstraßen (RASt) relevant. Die dort behandelten Abwägungsüberlegungen orientieren sich an verkehrlichen Größen, umfeldbezogenen Merkmalen,
% wirtschaftlichen Kriterien und raumordnerischen oder städtebaulichen Vorgaben. Die Richtlinien regeln auch grundlegend die entwurfstechnische und betriebliche Ausbildung von Kreisverkehren.
%
In Germany, there is no law stipulating the exact construction of roundabouts.
Instead, the elements of the rural roads and city streets are dealt with in Directives for the Design of rural roads \cite{ral13}
and the Directives for the Design of Urban Roads \cite{rast06}. These guidelines are also relevant to the choice of a convenient junction type when linking roads.
The considerations discussed there are based on traffic variables, area-related characteristics, economic criteria and spatial planning or urban planning requirements. 
The guidelines also regulate the basic design and operational formation of roundabouts.
The  Directives for the Design of Urban Roads \cite{rast06} are relevant for this dispute. Since the access the RASt ist limited, most of the information is coming from
\cite{man06} whereupon RASt is based on.
\subsection{Elements of a Roundabout}

\begin{figure}[!ht]
%%\begin{center}
\caption{Definition of individual design elements and dimensions of a roundabout \cite{man06}}
\includegraphics[width=0.7\textwidth]{bilder/kreisverkehr.png} %70% der Textbreite
\label{roundabout_parts}
%%\end{center}
\end{figure}


\begin{defi}[roundabout island]
The roundabout island is the constructional area in the middle of the roundabout, which is surrounded by vehicles.
For miniature roundabouts, the roundabout island is crossable. \cite{man06}
\end{defi}

\begin{defi}[circular path]
%Die Kreisfahrbahn ist die Fahrbahn, die zum Umfahren der Kreisinsel
%dient. Ein ggf. vorhandener Innenring ist verkehrsrechtlich nicht Be-
%standteil der Kreisfahrbahn (VwV-StVO zu §9a V., Rn. 5).

%%TODO
The circular path is the road that serves to drive the roundabout island. An inner ring, if present, is not part of the circular path (VwV-StVO zu §9a V., Rn. 5). \cite{man06}
\end{defi}

\begin{defi}[circular ring with ($B_K$)]
% Die bauliche Breite umfasst die Kreisfahrbahn und einen ggf. gepflasterten
% Innenring. Sie ist abhängig vom Außendurchmesser und der angestrebten
% Verkehrsführung (ein- oder zweistreifig). Die Randstreifenbreite orientiert
% sich an der maßgebenden durchgehenden Fahrbahn.
The structural width includes the circular track and a paved inner ring, if any. It is dependent on the outer diameter and the desired traffic routeing (one or two lanes). 
The edge strip width is oriented on the relevant continuous roadway. \cite{man06}
\end{defi}

\begin{defi}[outer diameter ($D$)]
The outer diameter is measured at the outer edge of the circular ring. It is the essential measure for describing the size of the roundabout. \cite{man06}
\end{defi}

\begin{defi}[inner diameter ($D_I$)]
The inner diameter is the diameter of the roundabout island. \cite{man06}
\end{defi}

\begin{defi}[road divider]
% Der Fahrbahnteiler ist die baulich ausgeführte Insel zwischen Kreisausfahrt
% und -zufahrt einer angeschlossenen Straße. Er dient der Trennung der
% Kreisaus- und -zufahrten, der Führung des Verkehrs sowie den Fußgängern
% und Radfahrern als Überquerungshilfe.
%
The road divider is the structurally designed island between the circular exit
and circular driveway. It serves to separate the circular exit and circular driveway, the management of the traffic, as well as the pedestrians and cyclists as cross-bordering aid. \cite{man06}
\end{defi}

\begin{defi}[lane width of the circular driveway ($B_Z$) and circular exit ($B_A$)]
% Die Breite der Kreiszufahrt und Ausfahrt wird am Beginn der Eckausrundung gemessen.
The width of the circular driveway and exit is measured at the beginning of the corner. \cite{man06}
\end{defi}

\begin{defi}[Corner rounding radius ($R_Z$ and $R_A$) ]
% Dies ist der Radius der Ausrundung am rechten Fahrbahnrand zwischen 
% der Kreiszufahrt und der Kreisfahrbahn. Bei einem Korbbogen mit einer
% Radienfolge aus drei unterschiedlichen Radien ist RZ der Radius R2 des
% mittleren Bogens. Bei der Ausbildung des Fahrbahnrandes als Schleppkurve ist RZ der kleinste Radius des Fahrbahnrandes.
% 
This is the radius of the rounding at the right edge of the road between the circular driveway and the circular path.
For a elliptical arch with a radius sequence of three different radii, $R_Z$ is the radius $R_2$ of the central arc.
When the road edge is formed as a tractrix, $R_Z$ is the smallest radius of the road edge. \cite{man06}
\end{defi}

\subsection{Types of Roundabouts}
There are several types of roundabouts, which are differentiated by the different application criteria and the partly different design principles according to the situation inside and outside built areas.
Furthermore, a division is made as a function of its size. \cite{man06}


\subsubsection{Mini Roundabout}

\begin{figure}[!ht]
%\begin{center}
\caption{Mini Roundabout \cite{man06}}
\includegraphics[width=0.5\textwidth]{bilder/mini_roundabout.png} %70% der Textbreite
\label{roundabout_mini}
%\end{center}
\end{figure}

Within built-up areas, smaller outer diameters are possible under certain conditions.
These roundabouts are called mini roundabout. The roundabout island must then be capable of being passed over.
The outer diameter should be at least 13 m, so that the circular island does not become too small.
Larger outer diameters make driving easier. Outer diameters of more than 22m, however, do not offer any transport advantages.
From an outside diameter of about 22 m, therefore, the installation of a small roundabout with 26 m is generally more convenient.
Bypasses are generally not required in the areas where mini roundabout can be used.


\subsubsection{Small Roundabout}

\begin{figure}[!ht]
%\begin{center}
\caption{Small Roundabout \cite{man06}}
\includegraphics[width=0.5\textwidth]{bilder/small_roundabout.png} %70% der Textbreite
\label{roundabout_small}
%\end{center}
\end{figure}

The small roundabout has a single lane circular path and single lane circular driveways and exits. The roundabout island is not passable.
The outer diameter must be at least 26 m. Bypasses can be set up for driving geometric reasons or to increase performance.


\subsubsection{Two-lane Passable Roundabout}

\begin{figure}[!ht]
%\begin{center}
\caption{Two-lane Passable Roundabout \cite{man06}}
\includegraphics[width=0.5\textwidth]{bilder/twolaned_roundabout.png} %70% der Textbreite
\label{roundabout_twolaned}
%\end{center}
\end{figure}

% Reicht die Kapazität des Kleinen Kreisverkehrs nicht aus und kann diese nicht durch die Anlage von Bypässen sicher gestellt werden,
% kann die Kreisfahrbahn eines Kleinen Kreisverkehrs zweistreifig befahrbar ausgebildet werden.
% An einem solchen Kreisverkehr ist die Kreisfahrbahn so breit, dass Pkw im Kreis nebeneinander fahren können.
% Wird eine weitere Erhöhung der Kapazität erforderlich, können einzelne Kreiszufahrten ebenfalls zweistreifig ausgeführt werden,
% wenn Fußgänger und Radfahrer regelmäßig nicht zu berücksichtigen sind. Kreisausfahrten werden aus Sicherheitsgründen immer einstreifig ausgeführt.
% Aus geometrischen Gründen muss der Außendurchmesser bei zweistreifiger Befahrbarkeit mindestens 40 m betragen.
%
If the capacity of the small roundabout is not sufficient and can not be ensured by the installation of bypasses,
the circular path of a small roundabout can be designed to be two-lane driveable.
At such a roundabout, the circular path is so wide that cars can travel side by side in a circle.
If a further increase in the capacity is required, individual circular driveway can also be carried out in two lanes, if pedestrians and cyclists are not to be considered regularly.
For safety reasons, circular exits are always carried out in single lanes.
For geometrical reasons, the outer diameter must be at least 40 m for two-laned accessibility.


\subsubsection{Large Roundabout}


\begin{figure}[!ht]
%\begin{center}
\caption{Large Roundabout \cite{man06}}
\includegraphics[width=0.5\textwidth]{bilder/large_roundabout.png} %70% der Textbreite
\label{roundabout_large}
%\end{center}
\end{figure}

%Große Kreisverkehre mit zwei oder mehreren durch Markierungen gekennzeichnete Fahrstreifen auf der Kreisfahrbahn sollen bei enger
%Abstimmung zwischen Knotenpunktentwurf und Verkehrssteuerung nur mit Lichtsignalanlage betrieben wer den.
Large Roundabouts with two or more lanes marked by markers on the circular path should be operated with a light signaling system only,
if the nodal point design and traffic control are closely coordinated.


\section{Middleware OpenDAVINCI}
Autonome Software ist typischer weise ein verteiletes System, auf heutigen Fahrzeugen basiert dieses System auf ECUs und Bussystemen wie CAN,LIN.
Verteilete Sysoftware vereinfacht es komplexe Komponenten innheralb des Systems zu integreieren. Im bereich des Autonomen Fahrens iust der Historische aufbau von Fahrzeugen
mit ECU'S und can jedoch nicht optimal. Um die vielen benötigekten Komponenten zu handhaben, ist es von vorteil Komponenten auch innerhalb eines ECU's bzw einer Recheneinheit
zu entkoppeln. Für diesen Zweck gibt es beireits mehrer middelwares die unteranderm die Kommunikation innerhalb der Komponenten handhaben und abstrahieren.
m Rahmen des Copplar Projekets, wird hier die OpenDaVINCI middleware genutzt. OpenDaVINCI ist eine echtzeitfähige laufzeitumgebeung konzipiert für Autonome Fahrzeuge.
OpenDaVINCI basiert auf Hesperia \cite{Berger2010}. Die Kommunikation zwischen den Komponenten basiert in OpenDaVINCI UDP Multicast, welches eine Echtzeitfähige Kommunikation zwischen
den Komponenten Ermöglicht \cite{Kurose2013}. Für die Kommunikation bietet OpenDaVINCI Time-triggered sender und Data-triggered receiver an, von welchem in folgenden der Data-triggered receiver
für die Anbindung der Software genutzt wird. Weiterhin bietet OpenDaVINCI viele weiter Funktionalitäten die das Handling von World Geodetic System 1984 (WGS84) Korridinaten an, welches für die Umwandlung
von GPS koordinaten in lokale kartesische genutzt werden kann. Dazu ist die Angabe einer referenz GPS postion nötig, welche um den Berechnugsfehler klein zu halten,
nicht zu weit entfernt sein sollte.


\section{Test Platform}
Die in dieser Abreit genutze Testplatform ist ein Volvo XC90 (2015) SUV. Dies Testpaltform ist mit vielen Sensoren zur Umfeldwarnehmung ausgestattet.
Dazu zählen fünf Radar Sensoren, rund um das Fahrzeug. Wobei das Front Radar über eine Größere Reichweite verfügt. Sowie eine
Stereo Kamera und ein Velodyne VLP-16 LiDAR. Die Anordnung der Sensoren kann \cref{platform} entnommen werden.


\begin{figure}[!ht]
%\begin{center}
\caption{Test Platform}
\includegraphics[width=\columnwidth]{sensors.pdf}
\label{platform}
%\end{center}
\end{figure}


Zusätlich zur Fahrzeuginternen Sensorik (Odometer, Interialsensorik) ist im Fahrzeug ein Applanix POS LV verbaut. Zu Zeitpunkt des Verfassens dieser Arbeit war es leider nicht möglich auf
die Radarsensoren und die Stereokamera zuzugreifen. Daher werden im folgenden lediglich der Velodyne Lidar und das Applanix System genauer beschrieben.

\subsection{Velodyne VLP-16 LiDAR}
Der Velodyne VLP-16 ist ein 360 Grad 3D Laserscannermit einer Rotationsgeschwindigkeit von 5 bis 20 Umdrehungen pro Sekunde. Er bietet ein vertikales FOV von 30 Grad, bei 2 Grad Auflösung.
Mit einer Reichweite von 100m kann er einen Umkreis von 200m Durchmesser abdecken. Weiterhin kann der VLP-16 mit dem Applanix POS LV syncronisiert werden, was eine jitterarme Zeimessung ermöglicht.
Eine weiter Funktion des Velodyne Sensors, ist das er auf verschiedene Messimpulse reagieren kann. Durch die Auswertung des letzten Impulses statt des Stärksten Impulses ist es Möglich durch Transparent Objekte zu sehen.
Das ermöglicht uns im späteren Verlauf die Breite des Fahrzeues zu ermitteln, da der Velodyne durch die Glasfenster des Fahrzeues blicken kann.
Bei einer eingestellten Geschwindigkeit von 10Hz liefert der VLP-16 eine Auflösung von 0.2 Grad bei einer Abreichung von +-3cm. Der VLP-16 ist mittig auf dem Dach des XC90 moniert, um eine möglichst hohe Positionierung
zu erreichen, die eine Rundumsicht umd Das Fahrzeug zu erreichen. Zu beachten ist, das diese Ausrichtung für den Sensor denkbar ungünstig ist, da der Sensor ein vertikales
Sichtfeld von -15 bis +15 Grad hat. Dadurch sind nachezu alle messungen über Null grad quasi nutzlos. Der blick auf die Herstellerseite
\footnote{\url{http://velodynelidar.com/vlp-16.html} (03/09/2017)}
verrät, das Der VLP-16 aunteranderem auf die verwendung mit Drohnen hin konstuiert wurde, während der Größere HDL64E
\footnote{\url{http://velodynelidar.com/hdl-64e.html} (03/09/2017)}
explizit für den Urbanen Automotivebereich beworben wird, und über ein Sichtfeld von +2 bis -24.9 Grad verfügt. Die dabei entstehenden Probleme werden später diskutiert.



\subsection{Applanix POS LV}
Das POS LV ist ein kompaktes Positions- und Orientierungssystem. Es Offeriert stabile, zuverlässige und reproduzierbare Positionierungslösungen für landgestützte Fahrzeuganwendungen.
Das POS LV liefert dabei eine Inertialsensork und Odometrie gestützte Positionsmessung mit einer Genauigkeit von bis zu 0.3m (bis zu 0.035m bei verwendung von der der RTK - Korrektur).
Im weiteren Verlauf wird außerdem das vom POS LV gelieferte Heading genutzt, welches eine Genauigkeit von 0.2 Grad liefert. Auch nach ausfall des GPS-Signals kann das POS-LV durch sein
Odeomerter und der Inertialsensork eine Position liefern. Diese wird jedoch über die Zeit schlechter, so das 60Sek nach Ausfall des GPS-Signals nich eine Genauigkeit von 2.51m erwartet
werden kann.\cite{manAP}








\chapter{Methodology}

\todo{siehe jens BA, dashier ist zu kurz}
% Gefragt wird hier nach den Kriterien dafür, welche Methode für eine bestimmte Art der Anwendung geeignet ist,
% warum eine bestimmte Methode angewandt werden muss oder angewendet wird und keine andere.
% Verständnisfragen zum methodischen Weg werden hier geklärt.
% Die Methodologie ist demnach eine Metawissenschaft und somit eine Teildisziplin der Wissenschaftstheorie.
% Demgegenüber bezeichnet Methodik das Methodenwissen des Praktikers oder des Wissenschaftlers.

Im vorigen Kapitel haben wir uns die Arten von Kreisverkehren und derren Komponenten angesehen.
\todo{kreisverkehre überabreiten}
Weiterhin haben wir die zur verfügung
stehende Testplatform und ihre Sensorik begutachtet. Dabei haben wir festgestellt, das für die Erkennung von Objekten in andren
Arbeiten häufig mehrere und teurere Sensoren kombiniert werden um ein Zuverlässiges erkennen von anderen Verkehrsteilnehmern zu gewährleisten.
\todo{beleg}
In der Research Questian eins haben wir festgehalten, das wir die verwendung eines günstigen VLP-16 Sensors in einem 
Komplexen Verkehrszenario, die Verkehrsbeobachtung eines Kreisverkehrs evaluieren wollen.

Dazu wird im folgenden ein Algorithms zum erkennen und Tracken von Objekten mit hilfe des Velodyne VLP-16 vorgeschlagen
und implementiert. Die Schwierigkeit besteht dabei in der Verwendung eines Einzigen und im vergleich günstigen
Umfeldsensors, welcher offensichtlich nicht als standalone Lösung für diesen Einsatzzweck entwickelt wurde.

Dieser Sensor bietet in seiner aktuellen Anwendung in dem für diese Arbeit relevanten Bereich eine vergleichsweise
geringe Auflösung. Daher schlagen viele in ähnlichen Projekten genutzte Gradienten basierende Algorythmen im Bereich 
der Segmentierung häufig fehl. \todo{beleg} Aus diesem Grund wird für die Segmentierung eine Groundplane basierender Algorithms implementiert.

Außerdem ist es Bauartbedingt in Kreisverkehren mit bebauten Mittelinseln und Mehrspurigen Kreisverkehren nötig
Fahrzeuge über ihren Messhorizont hinaus zu verfolgen, um ein sicheres einfahren in den Kreisverkehr zu gerwährleisten.
Zu diesem Zweck wird in Section \todo{section reference} ein Tracking und State Estimation Algorithms entwickelt welches dies gewähreisten soll.

Zur Evaluation dieser Algorythmen wurden mehrere Datensammlungen auf den nahe gelegenen schwedischen AstaZero \cref{astazero} Prüfgelände in Sandhult
durchgeführt, für alle nicht dort durchgeführten Expirimente werden in einer dafür erstellten Simuation durchgeführt. In dieser Simulation
wird ein Innerstätischer Kreiverkehr mit Fuß un Radweg nachgebaut, welche der Kreisverkehr auf AstaZero nicht bieten kann.

Die Evaluierung findet dabei von Hand anhand der grafisch aufbereiteten Messdaten statt. Dabei wird besonders auf False-Negativ
und False Positiv erkannte Hindernisse eingegangen. Grobe Außreißer bei der Position oder Orientierung dr Objekte werden ebenfalls verkmerkt.

Zur Evaluation der Handbarkeit des Kreisverkehrs wird außerdem eine Statemachine Implementiert welche das Fahrzeug Sicher und Unfallfrei
durch den Kreiverker bewegen soll. Dazu wird die Simulation über einen längeren Zeitraum beobachtet, und die Anzahl der eventuellen
Kollisionen notiert.

\begin{figure}[!ht]
%\begin{center}
\caption{AstaZero Proving Ground\\ \url{http://www.astazero.com/wp-content/uploads/2016/09/\%C3\%96versiktsskiss_mod.pdf} }
\includegraphics[width=\textwidth]{bilder/AstaZero.pdf}
\label{astazero}
%\end{center}
\end{figure}

% \section{Selection of Sensors}
% \section{Selection of Algorithms}
% \section{Simulation Enviroment}


\chapter{Research}
% \todo[inline]{The original todo note withouth changed colours.\newline Here's another line.}
% \unsure{Is this correct?}\unsure{I'm unsure about also!}
% \change{Change this!}
% \info{This can help me in chapter seven!}
% \improvement{This really needs to be improved!\newline\newline What was I thinking?!}

\section{Objekt Detection}

\subsection{Ground Removal}
Um in einer PointCloud Ojekte zu erkennen ist es nötig, zu wissen, welche Messungen zu Boden und welche zu Objekten gehören. Es gibt viele Möglichkeiten dieses
zu erreichen. Die Naivste Methode ist das entfernen, der Bodenplatte anhand ihrer Z-Koordinate. Diese Mehode hat allerdings viele Nachteile, zum einen muss
der LIDAR Sensor exakt gerade auf em Fahrzeug angebracht werden, zum anderen muss das Fahrzeug ein sehr steifes Fahrwerk haben, um eventuelle Neigungen des Sensors zu verhindern.
Weiterhin erlaubt dies ausschließlich die Entfernung von Palanaren Grundflächen, alo flache nicht hügelige Untergründe. Eine weitere Verbreitete Methode ist das 
Entfernen der Bodenplatte auf basis eines Statistischer mittelwertes \cite{Zhang}.  Diese Methode benötigt allerdings auch eine Kalibireirung der Sensorabstandes zum Boden.
Und die Bestimmung weiterer Schwellwerte, welche umgebungsabhäng sind. Die Votreile beider Methoden sind ihre gering nötige rechenleistung und laufzeit O(n).
Bessere Methoden wie Gradientenbasierende explansions algrythmenm benötigen einen Startpunkt der als Bodenplatte identifiziert werden kann.
Eine weitere Möglichkeit ist die Beschreibung von Objekten als Konvexe Objekte \cite{5164280}, die ebenfalls auf Basis der Gradienten beschrieben werden kann.
Vorteil dieser Methode ist das keine Initiale Position für die Bodenplatte benötigt wird.

Für unseren Anwendungsfall mit dem Velodyne VLP-16 besteht das Problem darin, dass die Auflösung des Sensor in der Höhe sehr gering ist. Abhängig von der Entfernung des 
Fahrzeuges innerhalb der benötigten Reichweite fallen nur zwei Lagen auf die Testfahrzeuge, wesshalb Gradientenbasierende Methoden hier zuverlässig versagen. \todo{beleg} Da die Gradienten zu klein sind und
die verkelinerung der nötigen Thresholds zu haufigen false Postitives führt. Die Methode des Statistischen Mittelwertes und die Methode auf basis der Z-Koordinate,
leiden am Fahrwerk des Volvo XC90 SUV. Die Höhe das Fahrzeuges ändert sich aunteranderem durch veränderung des Fahrprofiles (Sport/Eco, etc.) um mehrere Zentimeter.
Auch leicht erhöhte Geschwindigekiten im Kreisverkehr (ca 30 km/h) führen zu einer deutlichen Seitenneigung des Fahrzeuges. Darum wird nun eine weitere Methode vorgeschlagen.
Die Erkennung einer Grundfläche in den Messdaten. 

Für die Erkennung des Bodens gehen wir von Folgenden Annahmen aus, die Straße lösst sich approximativ als Ebene im R3 darstellen. Weiterin ist die Grundfläche die niedrigste
Fläche im gesuchten Bereich. Daher wird im ersten schritt der in Polarkoordinaten vorliegende Datensatz in  in 30 Tortenstück förmige Segmente geteilt.
Aus diesem Tortenstück werden dann jeweils vorne und hinten zwei Segmente [\cref{segments}] ausgewählt, welche nicht beachbart sind. Die Auswahl der Segmente folgt aus der Annhame,
dass sich die Straße vor, bzw hinter dem Fahrzeug befindet. Zukünfitg könnte die Auswahl der Segmente auch mit hilfe des Fahrzeuglenkwinkes optimiert werden. Oder der Gültige Bereich
von einer Lane Detection geliefert werden.

\begin{figure}[!ht]
%\begin{center}
\caption{LiDAR Segments}
\includegraphics[width=\textwidth]{bilder/segments.png}
\label{segments}
%\end{center}
\end{figure}

\todo{beschreiben warum hinten messwerte fehlen, in TestPlatform Kapietel}
Innerhalb dieser Tortenstücke wird dann eine Suche nach den 10 Messungen mit dem niedrigsten Z Wert gesucht. Die Suche beschränkt sich dabei auf die drei niedrigsten Lagen (-15,-13,-11 Grad),
da alle höheren Lagen zu wit weg währen. Die Einteilung in Segmente ist desshalb nötig um zu verhindern, dass alle Messerte in ein einziges lokales Minima laufen.
Aus diesem Vorgefilterten Messwerten werden nun für einen \ac{RANSAC} drei zufällig herausgesucht.
Aus diesm dei Punkten wird nun eine Ebene in der Hessischen Normalform gebildet ,was eine effiziente Distanzberechnung zu anderen Punken erlaubt. Danach sammeln wir alle weiten Punkte aus unseren Minima, anahnd eines Distanzkriteriums.
Danach wird aus der Ebene und den neu Gesammelten Punkte durch einen Planefitting Algorithms [\cref{subssec:planefitting}] eine neue Ebene und derren Fehler berechnet.
Der Fehler wird über die Summe der quadratischen Abstände aller Punkte zur Ebene berechnet.

Bevor wir die Ebene jedoch als eventueller Lösungskanidat hinzufügen wird geprüft ob sich die Ebene innerhalb von einem plausiblen Parameterbereich befindet.
Dazu zählt, dass die Entfernung der Ebene zwischen 1.9m und 2.2m bewegen sollte, dies entspricht in etwa der Montagehöhe des Velodyne Sensors.

Die Anzahl an Iterationen des \ac{RANSAC} ist auf 50 Begrenzt. Nach dem Durchlauf des \ac{RANSAC} werden alle Punkte in der Pointcloud anhand ihrer Distanz zur Ebene als
Groundflache makiert. Als threshold wurde hier expirimentel ein optimaler Wert von 0.5m ermittelt.


\subsubsection{Planefitting}
\label{subssec:planefitting}
Zum Planefitting einer be ne wird üblicherweise eine \ac{SVD}  \cite{Nurunnabi2012,Ram2007,Soderkvist2009}.
SVD hat eine Komplexität von $\mathcal{O}(\min\{mn^2, m^2n\}$ \cite{Holmes2007}, da das Planefitting innehalb 
des \ac{RANSAC} sehr häufig mit einer großen Anzahl an Punkten ausgeführt wird, führt das Ausführen des \ac{SVD} innhalb des \ac{RANSAC} zu einer sehr hohen laufzeit.
Deshalb wird an dieser Stelle ein \ac{LLSQ} Algorithus mit einigen optimierungen eingesetzt. Bei der verwendung des \ac{LLSQ} gilt es zu beachten,
dass nicht der abstand der Punkte zur eben optimiert wird, sondern der Abstand der Punkte zur Ebene entlang einer Achse (in unserem Fall der z Achse) siehe \cref{LLSQ_MIN}.
Das kann zu Problemen führen, wenn die Punkte weit gestreut, also weit von der Optimalen Ebene entfert sind. Da wir unsere Punkte innhalb des \ac{RANSAC} allerdings anhand eines 
Distanzkriteriums vorselektieren, stellt dies kein Problem dar.

\begin{figure}[!ht]
%\begin{center}
\caption{Linear least Squares (LLSQ)  \cite{LLSQ}}
\includesvg[width=0.5\textwidth]{Linear_least_squares_min}
\label{LLSQ_MIN}
%\end{center}
\end{figure}

Die Darstellung einer Ebene in Koordinatenform sieht wie folgt aus: $ a\vec{x} + b\vec{y} + c\vec{z} + d = 0 $. Da wir eine Ebene im R3 betrachten, ist dieses Gleichungsystem überbestimmt.
Da wir unsere Ebene in Richtung der Z-Achse optimieren wollen setzten wir Parameter c auf 1 und können unser Gleichungssystem nun einfach nach z auflösen: $a\vec{x} + b\vec{y} + d = -\vec{z}$.
Die Vektoren $\vec{x},\vec{y},\vec{z}$ stellen dabei die zu fittenden Punkte dar.
In Matrixschreibweise:

\begin{align*}
X \vec{\beta} &= \vec{z}\\
\begin{bmatrix}
x_0 & y_0 & 1 \\
x_1 & y_1 & 1 \\
 & \dots & \\
x_n & y_n & 1 
\end{bmatrix} 
\begin{bmatrix}
a \\
b \\
d 
\end{bmatrix}
&= 
\begin{bmatrix}
-z_0 \\
-z_1 \\
\dots \\
-z_n 
\end{bmatrix} 
\end{align*}

Dieses Sytem hat üblicherweise keine Lösung, unser eigentliches Ziel ist jeoch auch nicht extakte lösungen für $\vec\beta$ zu finden sondern eine gute näherung $\hat{\beta}$ dafür:

\begin{align*}
\hat{\beta} = \min{(|| \vec{z} - X\vec{\beta} ||^2)}
\end{align*}

Das können wir tun indem wir unsere Gleichung mit der Transponierten unserer Punktmatrix $X$ mulltiplizieren:
\begin{align*}
(X^TX) \hat{\beta} &= X^T \vec{z}\\
\begin{bmatrix}
x_0 & x_1 & \dots & x_n \\
y_0 & y_1 & \dots & y_n \\
1 & 1 & \dots & 1  
\end{bmatrix} 
\begin{bmatrix}
x_0 & y_0 & 1 \\
x_1 & y_1 & 1 \\
 & \dots & \\
x_n & y_n & 1 
\end{bmatrix} 
\begin{bmatrix}
a \\
b \\
d 
\end{bmatrix} 
 &= 
\begin{bmatrix}
x_0 & x_1 & \dots & x_n \\
y_0 & y_1 & \dots & y_n \\
1 & 1 & \dots & 1  
\end{bmatrix} 
\begin{bmatrix}
-z_0 \\
-z_1 \\
\dots \\
-z_n 
\end{bmatrix} 
\end{align*}

Dieses Gleichungssystem könne man nun mit der Berechnung der Inverse von $(X^TX)$ auflösen. Da die Berechnung von Inversematritzen mit $\mathcal{O}(n^3)$ ebenfalls aufwändig ist,
nun ein weiterer Trick um rechenleistung zu sparen.
Nach dem Multiplizieren der Transponierten erhalten wir:
\begin{align*}
\begin{bmatrix}
\sum x_i x_i & \sum x_i y_i & \sum x_i \\
\sum y_i x_i & \sum y_i y_i & \sum y_i \\
\sum x_i & \sum y_i & N
\end{bmatrix} 
\begin{bmatrix}
a \\
b \\
d 
\end{bmatrix} 
 = 
\begin{bmatrix}
\sum x_i z_i \\
\sum y_i z_I \\
\sum z_i 
\end{bmatrix} 
\end{align*}

Gut zu sehen sind hier die Summen in den Randbereichen der Matrix X und dem Vektor $\vec{z}$. Diese können wir auf Null setzten, wenn wir alle Punkte relativ zum Mittelwert-Punkt
aller Punkte definieren, also $P_i = P_i - \overline{P}$. Nun erhalten wir:

\begin{align*}
\begin{bmatrix}
\sum x_i x_i & \sum x_i y_i & 0 \\
\sum y_i x_i & \sum y_i y_i & 0 \\
0 & 0 & N
\end{bmatrix} 
\begin{bmatrix}
a \\
b \\
d 
\end{bmatrix} 
 = 
\begin{bmatrix}
\sum x_i z_i \\
\sum y_i z_I \\
0 
\end{bmatrix} 
\end{align*}

Nun können wir d ebenfalls auf Null setzten, denn wenn alle unsere Punkte relativ zum Mittelwert-Punkt sind, dann läuft auch unsere Ebene immer durch diesen Punkt. Daher können wir 
nun eine komplette Dimension streichen:

\begin{align*}
\begin{bmatrix}
\sum x_i x_i & \sum x_i y_i \\
\sum y_i x_i & \sum y_i y_i
\end{bmatrix} 
\begin{bmatrix}
a \\
b 
\end{bmatrix} 
 = 
\begin{bmatrix}
\sum x_i z_i \\
\sum y_i z_i
\end{bmatrix} 
\end{align*}

Das Gleichungssystem können wir nun einfach mit der Cramer's rule lösen
\begin{align*}
D &= \sum x_i x_i \cdot \sum y_i y_i - \sum x_i y_i \cdot \sum x_i y_i \\
a &= \frac{\sum y_i z_i \cdot \sum x_i y_i - \sum x_i z_i \cdot \sum y_i y_i }{D}\\
b &= \frac{\sum x_i y_i \cdot \sum x_i z_i - \sum x_i x_i \cdot \sum y_i z_i }{D}\\
\vec{n} &= [a, b, 1]^T
\end{align*}
Dabei gibt es zu beachten, dass die Determinante nicht Null oder nahe Null sein darf.
Da der winkel zwischen dem Fahrzeug und der Ebene jedoch immer nahe 90 Grad liegt, ist die Determinante typischweise sehr groß. 
Sollte die Determinante doch nahe 0 (nicht gleich 0) sein, wird die Berechnung trotzudem durchgeführt, da dies auch zu einem 
großen Fehler im Fitting führt. Dies ist an dieser Stelle erwünscht, da der \ac{RANSAC} ungeültige Ebenen anhand des Fehlers ausssortiert.
Ist die Determinante extakt Null, wird die Berechnungstatdessen mit einem kleinen Wert für D fortgesetzt.

Aus dem Normalenvektor $\vec{n}$ und dem Mittelwert-Punkt $\overline{P}$ können wir nun wieder die Hessische Normalenvektor bestimmen.

Letztendlich haben wir so den Algorithms von $\mathcal{O}(m^2n) $ auf $\mathcal{O}(n) $ runterbrechen können.

\subsection{Clustering}
In aktuellen Abreiten mit 3D-LIDAR Daten werden de Daten haufig als erstes in eine Heightmap projeziert \cite{Zhang,Himmelsbach2009,Li2016}.
Danach werden direckt benachbarte Messungen mit ähnlichen Messwerten zusammengefasst. Alternativ werden die Messungen auch anhand eines Distanzkritterums 
zusammengefasst. Erstere Methode hate den Nachteil, dass einzelne Ausreißer dazu führen, das das Objekt in mehrer Cluster zerfällt.
Letztere wird meißt mit einem KD-Tree oder einer ähnlichen Datenstrucktur kombiniert, welche typischerweise hohe Kosten für die Erstellung verursachen.
Da der Baum nach jeder 360 Grad messung neu Aufgebaut werden muss ist das Problematisch

Hier wird eine Methode vorgeschlagen welche die Vorteile beider Methoden kombiniert. Dauzu ist es nötig zu wissen, wie die Daten von der OpenDAVINCI 
Middleware geliefert werden. Das OpenDAVINCI auf der Übertragung der Daten mit UDP Multicast setzt, werden die Daten in einer Kompakten form übertragen, welche in einen einzigen
UDP Frame passt.\\
\begin{center}
\begin{tikzpicture} 
\umlclass{CompactPointCloud}{
startAzimuth : float \\
endAzimuth : float \\
entriesPerAzimuth : uint32 \\
distances : byte[]}
{getStartAzimuth : float\\
\dots}
\end{tikzpicture}
\end{center}
Dabei wird von einer konstanten Drehrate des Sensors ausgegangen, was in einer äquidistanten der Messwerte resultiert. Die Anzahl der Messungen pro Azimuth
wird in entriesPerAzimuth festgehalten und entspricht für den Velodyne VLP16 16. Um nun an die Eigentlichen Messwerte zu kommen müssen jeweils zwei distance Werte zu
einem Unsigned 16Bit Integer umgewandelt werden, welcher dann die Messung in cm enthält. Jeweils 16 dieser Werte ergeben dann einen Messframe in dem der Polarwinkel
auf einen Bereich zwischen -15 und +15 abgebildtet werden muss. Nachdem die sphärische Daten wiederhergestellt wurden, werden diese In Kartesische umgewandelt und
in eine Punkt Datenstruktur gespeichert. 

\begin{center}
\begin{tikzpicture} 
\umlclass{Point}{
azimuth : float \\
measurement : float \\
visited : bool \\
isGround : bool \\
point : vector3f \\
}
{getAzimuth : float\\
\dots}
\end{tikzpicture}
\end{center}

Diese wird wiedrrum in ein Statisches 2 Dimensionales Array Gespeichert: Points[2000][16]. Die Reihenfolge der Daten wird dabei beibehalten.
Diese Datenstruktur stellt nun im weiteren verlauf unsere Pointcloud dar.

\todo{warum DBSCAN mit Noise!}
Auf dieser Basis wird nun ein \ac{DBSCAN} \cite{DBSCAN} ausgeführt. Der \ac{DBSCAN} Algorithms hat dabei folgende Vorteile.
Im Gegensatz beispielsweise zum K-Means-Algorithmus, muss nicht im vornherein bekannt sein, wie viele Cluster existieren. Der Algorithmus kann Cluster beliebiger Form 
(z.B. nicht nur kugelförmige) erkennen. Das macht den \ac{DBSCAN} damit für uns zu optimalen Kandidaten. \ac{DBSCAN} selbst ist von linearer Komplexität.
Die meiste Rechenzeit wird jedoch überlichweise durch die Nachbarschafts berechnung verursacht. Genau hier setzen wir an, anstatt der Bereichsanfrage über eine Baum-Struktur
nutzten wir aus, dass Messungen in einer kleinen Nachbarschaft einen ähnlichen Azimuth Winkel haben. Dazu untersuchen wir für jeden Messwert jeweils zwei weitere Einträge nach links
und rechts in unserem Array. Effektiv müssen wir daher $5 \cdot 16 = 80$ werte Überprüfen. Die Laufzeit der Bereichsanfrage kann deshalb in linearer Komplexität durchgeführt
werden. Alle Messwerte die Zuvor als Grund Klassifiziert wurden, werden bei der Berchnung übersprungen, zusätzlich entfällt der Aufbau eines KD-Trees. 


\subsection{Tracking}

\todo{datenstrucktur Onjekte hinzufügen, weil positionen und attribute verwirrend}
Das Tracking ist in zwei Abschnitte unterteilt. Dem Tracking der Cluster vom \ac{DBSCAN} und dem Erstellen und Tracken von Hindernissen.

\subsubsection{Cluster Tracking}
Für das Tracking der Cluster nehmen wir an, dass sich Objekte von Zeitschritt zu Zeitschritt nur geringefügig bewegen, weiterhin
ändert sich die Form der Cluster ebenfalls nur leicht. Das ist wichtig, da die Position eine Clusters durch einen Mittelwertpunkt definiert ist.
Das Tracking wird im R2 durchgeführt. Im Initialen Schritt wird jeden Cluster eine aufsteigende ID zugerordnet.
In jedem weiteren Schritt wird jedem neuen Cluster die ID des alten Clusters zugeordnet welcher über die Zeit hinweg die geringste Entfernung aufweißt.
Fur diese Entfernung gibt es eine großzügige obere Schranke von 3m, Cluster die nicht innerhalb in dieser Grenze sind erhalten eine neue ID.
Das führt dazu, dass mehreren Cluster die selbe ID zugeordnet werden kann, das ist wichtig, da Objekte manchmal in mehrere Cluster zerfallen.

\subsubsection{Object Tracking}
Basis für das Objekt Tracking sind die zuvor gtrackten Cluster. Im Initialen Schritt werden aus allen Cluster mit der Selben IDs Objekte gebildet.
In jeden weiteren Schritt werden Alle Cluster mit der zuvor gleichen ID zum updaten der Objekte genutzt. Aus Clustern mit neuen IDs werden neue Objekte gebildet.

Der vorerst wichtigste Schritt beim Updatevorgang ist die berechnung der Bewegungsrichtung eines Objektes, da folgende Berechnungen auf dieser basieren.
Bei der Berechnung der Bewegungsrichtung ist zu beachten, dass die Bewegung des eigenen Fahrzeuges herausgerechnet werden muss.
Dazu werden die Positionsdaten des Applanix POS-LV genutzt. Da sowohl die Positionsdaten des Applanix Systems, als auch die Erkannte Postition des Fahrzeuges Fehlerbehaftet sind,
wird die Bewegungrichtung nur bei einer minimalen Bewegung von 2m geupdatet.

\begin{align*}
\Delta x &= P_x(t) - P_x(t_{-2m}) + \Delta C_x\\
\Delta y &= P_y(t) - P_y(t_{-2m}) + \Delta C_y\\
\theta &= \atantwo(\Delta y,\Delta x)
\end{align*}

mit $P$ - Postition des Objekts, $C$ - Position des eigenen Fahrzeuges

Das Ergebnis kann in \cref{obst_rot} betrachtet werden. Gut zu erkennen ist, dass die Bewegungsrichtung (Pfeil) nicht mit der Ausrichtung des Objekte (schwarz) übereinstimmt,
die Boundingbox jedoch korrekt ausgerichtet ist. Wie diese Berechnung zustande kommt wird im folgenden geklärt.

\begin{figure}[!ht]
%\begin{center}
\caption{Obstacle Movement}
\includegraphics[width=0.5\textwidth]{bilder/obst_rot.png}
\label{obst_rot}
%\end{center}
\end{figure}

Basierend auf der Bewegungrichtung wird nun die Ausrichtung des Fahrzeiges berechnet. Dazu werden alle dem Objekt zugewiesenden Cluster zusammengefasst und um $-\theta$ gedreht.
Danach wird das Objekt wird in 3 gleich große Segmente unterteilt (\cref{obst_devide}).

\begin{figure}[!ht]
%\begin{center}
\caption{Obstacle Cutting}
\includegraphics[width=0.5\textwidth]{bilder/obst_devide.png}
\label{obst_devide}
%\end{center}
\end{figure}

Weiterhin wird bestimmt ob sich das Objekt oberhalb oder unterhalb der x-Achse befindet. Dies ist wichtig, da wir wissen müssen, welche seite des Objekts wir messen.
Befindet sich das Objekt aso unterhalb der x-Achse wird im nächsten Schritt der y-Wert maximiert, befindet es sich oberhalb, wird er minimiert.
Im folgenden gehen wir davon aus, dass sich das Objekt unter der x-Achse befindet. Deshalb maximieren wir nun im linken und rechten Segment des Geteilten
Hindernises die y-Werte. Die Unterteilung in 3 Segmente ist nötig um zu verhindern, das bei perfekt wagerechte ausgerichteten Objekt beide Maxima in den selben Punkt laufen.
Mit diesen Punkten $(\vec{R};\vec{L})$ wird nurn eine korrektur der Drehung des Objektes berechnet:

\begin{figure}[!ht]
%\begin{center}
\caption{$\theta$ - Correction}
\includegraphics[width=0.5\textwidth]{bilder/obst_devide_angle.png}
\label{obst_correction}
%\end{center}
\end{figure}

\begin{align*}
\Delta x &= R_x - L_x\\
\Delta y &= R_y - L_y\\
\theta_{correction} &= \atantwo(\Delta y,\Delta x)
\end{align*}

Nach Anwendung der Korrektur wird die größe des Hindernises berechnet. Dazu werden die maximalen und minimalen x und y Werte herangezogen.
Mit diesen Werten wird nun über die Zeit ein auf 0.5m gerundetes Histogram für die Länge und Breite des Hindernises aufgebaut.
Anhand diesem wird dann der warscheinlichste Wert ausgewählt. Dadurch ändert sich die Größe des Hindernisses zu begin häufiger,
bevor die Größe auf einen stabilen wert konvergiert.Da die größe des Objekts zu begin sehr klein sein kann, 
gibt es für beide Werte einen unteren Grenzwert. Messerte für ein Beispielobjekt sind in \cref{obst_hist} zu sehen.
\begin{figure}[!ht]
  %\addcontentsline{lof}{figure}{LaTeX--Strukturelemente}
%\begin{center}
\caption{Object Size Histogram}
\begin{tikzpicture}
\begin{axis}[ybar interval, ymax=366,ymin=0, minor y tick num = 3,
width=0.55\textwidth,
height=0.6\textwidth]
\addplot coordinates { (1.0, 29) (1.5, 58) (2.0, 366) (2.5, 35) (3.0, 3) (3.5, 1) };
\end{axis}
\begin{axis}[ybar interval, ymax=318,ymin=0, minor y tick num = 3,
width=0.55\textwidth,
height=0.6\textwidth,
at={(0.50\textwidth,0)}]
\addplot coordinates { (2.0, 7) (2.5, 6) (3.0, 9) (3.5, 2) (4.0, 64) (4.5, 84) (5.0, 318) (5.5, 3) (6.0, 2) (6.5, 2) };
\end{axis}
\node at (0.21\textwidth,0.5\textwidth) {width};
\node at (0.7\textwidth,0.5\textwidth) {length};
\end{tikzpicture}
\label{obst_hist}
%\end{center}
\end{figure}

Leicht zu sehen wird für das Object eine breite von 2m und eine Länge von 5m berechnet. Das Objektist in diesm Fall ein Volvo S60, welcher Außenmaße von ca. 1.9m und 4.6m hat,
womit die Abweichungen sich korrekt innerhalb, der Rundung der Werte befinden. Für die Nachfolgende filterung der Messwerte mit Hilfe eines Kalman filters, wurd nun
die Position des Fahrzeuges aus dem Mittelpunkt der Boundingbox bestimmt.

Die für die Berechnung der Bewegungsrichtung genutzte Position ist jedoch eine andere, da die so eben berechnete Position zu diesem Zeitpunkt noch nicht zur verfügung steht und
die Position kurz nach der Initialen erkennung durch die häufigen Größenändrungen sehr instabil ist. Deswegen wird als Position immer die maximale x-Koordinate der Cluster gesnutzt.
Diese Postition kann ebenfalls in \cref{obst_rot}, als grüner Punkt begutachtet werden.
Da $\theta$ im initialen Zeitschritt Null ist, entpricht dies der globalen Maximalen x-Koordinate des Clusters. Dies Führt dazu, das wir im Initialen Zeitschritt annhemen, dass sich das Objekt
in die positive x-Richtung bewegt. Bei Objekten bei denen das nicht der Fall ist, führt das zu einem kurzzeitigen oszillieren der Orientierung, welche sich jedoch über die Zeit schnell stabilisiert.


Der eine oder ander möge sich wundern warum die Boundingbox so aufwendig berechnet wurde. Eine Einfache Art und weise eine Boundingbox für die Objekte zu berechnen wäre die berechnung
der Minimum Boundingbox über die konvexe Hülle, so wie es in vielen Anderen Arbeiten gemacht wird \cite{Zhang,Himmelsbach2009}. Die Minimum Bounding box liefet jedoch unter umständen nicht das gewünschte
ergebnis, zum einen hat die immer nur die größe der aktuellen Messung und zum anderen kann sie eine falsche Orientierung liefern, wie in \cref{min_box} zu sehen.

\begin{figure}[!ht]
%\begin{center}
\caption{Error with minimum Boundingbox \cite{Himmelsbach2009}}
\includegraphics[width=\textwidth]{bilder/min_bound.png}
\label{min_box}
%\end{center}
\end{figure}

Da die orientierung und Postition der Objekte jedoch als eingang für den nachfolgenden Kalmanfilter genutzt wird, welcher sehr sensible auf falsch Orientierungen reagiert, wurde eben jeder Algorithmus
entwickelt.

\subsubsection{Object Confidence}
\todo{implementierung Konfidence wert auf 1 setzen und entfernen, wenn wert unter 1, oder objekte auf basis ihrer Position entfernen}
Um zu verhindern, dass kurzzeitig erkannte False Positives direckt als Objekt erkannt werden und damit die nachfolgende Logik beeinflusst, wird nun ein Confidence Wert eingeführt
Bevor ein erkanntes Objekt als gültig erachtet wird, muss dieses einen gewissen Confidence Wert erreichen. Der Initiale Konfidence Wert eines Objektes ist Null. Der Konfidence wert
wird um eins erhöht, wenn das Objekt in zwei aufeinader folgenden Zeitschritten getrackt werden kann und folgende bedingungen erfüllt: 

\begin{itemize}
\item Die breite des objkts muss kleiner seine als die länge des hindernisses zuzüglich 1.5m
\item Die länge des Hindernises muss kleiner sein als 10m
\item Die Breite des Hindernisses muss kliener sein als 4m
\end{itemize}
Ist eine dieser Bedingunggen nicht erfüllt, wird der Confidence Wert statdessen halbiert. Damit ein Objekt als gütig erachtet wird, muss es einen Confidence wert von 3 erreichen, sodass
Ein Objekt erst nach mindestens 3 drei Iterationen erkannt werden kann.



\subsection{Classification}
\todo{text zu listing anpssen}
\todo{implementierung Plausibilitätstest}
\todo{filterung in code entfernen, weil größe bereits gefiltert wird}
Die Klassifizierung der Objekte finedt anhand ihrer Größe statt, es findet keine Klassifizierung nach bewegten und unbewegten Objeten Stadt.
Unterschieden werden lediglich Fußgänger, Radfahrer, und Fahrzeuge, und Sonstige. Als Klassifizierungs kritterium wird die Größe der Onjekte genutzt.
Die Klassifizierung sieht wie folgt aus:
\begin{description}
\item[pedestrian:] length < 1.5 and width < 1.5
\item[cyclist:] length < 2 and width < 1.5
\item[car:] length < 10 and width < 4
\item[undefined:] length >= 10 and width >= 4
\end{description}

% Objekte die eine Größe von 2x1.5m überschreiten werden dabei als Fahrzeug Klassifiziert, Objekte die kleiner als 1.5x1.5m sind werden als Fußgänger klassifiziert. Alles dazwischen wird als Radfahrer klassifizierfiziert.
Weiterhin wird andhand der Geschwindigekeit eine Plausibilitätstest durchgeführt. So darf eine Fußgänger eine Geschwindigekeit von 10km/h nicht überschreiten und die Maximalgeschwindigkeit dür einen 
Radfahrer bertägt 30km/h. Da für Fahrzeuge keine sinnvolle geschwindigkeitsgrenze angenommen werden kann, wir an ihrer stelle die Änderung der Orientierung genutzt. Als Maximale Drehrate wird
eine Messung von 0.3 rads/sec aus \cite{Kelly1994} angenommen. Da der Wert eine Obere Grenze darstellen soll nehmen wir einen etwas höheren  Wert von  0.3 rads/sec an.


% \begin{lstlisting}[language=Python]
% if length < 1.5 and width < 1.5:
%     type = pedestrian
%     
% elif length < 2 and width < 1.5:
%   type = cyclist
%   
% elif length < 10 and width < 4:
%     type = car;
% else:
%   type = undefined
% \end{lstlisting}


\subsection{State Estimation}

Nachfolgende des Trackings wird auf den Erkannten Objekten eine Zustandschätzung durchgeführt. Dies ist nötig, da Objekte während der Fahrt verdenkt werden können,
sei es durch andere bewegte Objekte oder Gebaüde. Eine schätzung des Zustandes über den Erkennungshorizont hinaus erlaubt es uns eine Aussage über die Position von
Objekten zu treffe, welche im Moment nicht sichtbar sind. Weiterin erlaubt es auf einfache Weise zeitweise nicht erkannte Objekte wiederzuerkennen, also dem Objekt
die gleich ID zuzuweisen wie zuvor. 

Aus dem zuvor durchgführten Tracking können wir die Aktuelle Position, Geschwindigekeit, Drehung und Drehgeschwindigkeit erhalten. 
Für eine Zustandschätzung mit den für Fahrzeuge üblichen Bicycle Model fehlen und Angangen über den Radstand und Gewicht des Fahrzeues.
Daher müssen wir uns auf ein relativ Einfaches ``Constant Turn Rate and Velocity'' Model beschränken. Dies erlaubt es uns allerdings das gleiche
Model für alle Klassen von Objekten zu nutzten. Da dieses Model ebenfalls auf Fußgänger und Radfahre angewendet werden kann.

\subsubsection{Constant Turn Rate and Velocity Model}
Der Zustandvektor \cite{Schubert2008} des CTRV- Modells sieht wie folgt aus:
\begin{align*}
\vec{x}(t) &=
\begin{bmatrix}
x & y & \theta & v & \omega
\end{bmatrix}^T\\
x &\text{ - Y Axis}\\
y &\text{ - X Axis}\\
\theta &\text{ - Object Yaw Angle}\\
v &\text{ - Object Velocity}\\
\omega &\text{ - Yaw Rate}
\end{align*}

Die Dynamikmatrix erhalten wir durch eine nichtlinieare Zustandsübergang:

\begin{align*}
\vec{x}(t + T)=
\begin{bmatrix}
\frac{v}{\omega} (-\sin(\theta)) + \sin(T \omega + \theta) + x(t) \\
\frac{v}{\omega} (\cos(\theta)) - cos(T \omega + \theta) + y(t) \\
\omega T + \theta\\
v\\
\omega
\end{bmatrix} 
\end{align*}

% Da es sich beid er Dynamikmatrix um einen nichtlinearen Zustandsübergang handelt benötigen wir noch die Jacobi Matrix.
% Die Matrix wure mit Sympy generiert, und auf Grund ihrer Größe eingekürzt.
% 
% 
% \begin{align*}
% J(\vec{x})=
% \begin{bmatrix}
% 1 & 0 & \frac{v}{\omega} (-\cos(\theta)) + \cos(T \omega + \theta)& \dots & \dots \\
% %\frac{1}{\omega} (-\sin(\theta)) + \sin(T \omega + \theta)& multi-body simulation
% %\frac{Tv}{\omega} \cos(T \omega + \theta) - \frac{v}{\omega^2}(-\sin(\theta) + \sin(T\omega + \theta) 
% 0 & 1 & \frac{v}{\omega} (-\sin(\theta)) + \sin(T \omega + \theta)& \dots & \dots \\
% 0 & 0 & 1 & 0 & T\\
% 0 & 0 & 0 & 1 & 0\\
% 0 & 0 & 0 & 0 & 1
% \end{bmatrix} 
% \end{align*}

\subsubsection{Prediction}
Befindet sich ein Objekt innerhalb der Rechihweite des Velodyne Sensors und wird im Darauffolgenden Zeitschritt nicht erkannt,
wird der Prediktionsschritt des Kalman filter weiterhin ausgeführt. Dies geschieht solange die Unsicherheit der Position einen gewissen Schwellwert üerschreitet.
Sobald das Clustertracking dann ein neues Objekt detektiert, dem keine Bisher bekannte ID zugewiesen werden kann, wird die Postion mit allen Onjekten in der Prediktionsphase
abgeglichen. Befindet sich das neue Objekt nahe an der predizierten Position wird der Cluster dem Onjekt zugewiesen, und der Korrekturschritt des EKF durchgeführt.



\section{Simuation}
Die Simulationsumgebeung, welche verwendet wird ist VREP \footnote{\url{http://www.coppeliarobotics.com/}}. VREP wurde für verschiedenne Robottikanwendungen entwickelt.
VREP erlaubt es innerhalb eines Graphischen beliebige \ac{MKS}, basierend auf diversen Physikengines zu konstruieren. Weiterhin verfügt VREP bereits über
viele fertige Sensormodell, wie beispielsweise den Velodyne VLP-16. Die ganze Simulation kann über ein RemoteAPI Interface mit nahezu jerder Programmiersprache kommunizieren.

\begin{figure}[!ht]
%\begin{center}
\caption{VREP}
\includegraphics[width=\textwidth]{bilder/path.png}
\label{vrep}
%\end{center}
\end{figure}

\subsection{Simuation Scenario}
Für Das Simalations Scenario wurde eine einfacher kleiner Kresiverkehr mit einem Außendurchmesser von 26m designt, da dieser aud grund seiner größe das interessanteste Objekt ist.
Und im Innerstätischen Bereich jaufig anzutreffen ist. Zum testen der Onjektdetektion bietet dieser auf grund seiner bebaubaren mittelinsel gute möglichkeiten das Tracking zu testen.
Weiterhin ist es aufgrund seiner Größe möglich den kompletten Kreisverkehr zu überblicken. Das Ganze Senarion wurde aufgrund von limitierungen in Vrep um den Faktor 10 herunterscaliert.

Innhrhalb des Szenarosn bewegen sich innerhalb des Kreisverkehrs ein Fahrad, ein Fußgänder und ander 2 Fahrzeuge. Die Objekte bewegen sich dabei auf festgelegten Pfaden.
Dei Geschwindigekeit aller Verkehrsteilnehmer ist dabei auf den Typ angepasst. Der Füßgänger bewegt sichn mit für füßgänger übliche 5km/h.
Das Fahrrad mit 15km/h. Die beiden Autos bewegen sich mit Unterschiedlichen Greschwindigkeiten zwischen 25 und 35 Km/h. Um eine Kolision der Fahrzeuge zu vermeiden, sind beide 
Fahrzeuge an der vorderseite mit Distanzsensoren ausgestattet. Sollte dieser Sensor ein Fahrzeug erkennen, geht das Fahrzeug in eine  Adaptive cruise control modus und
passt sich dem vorderen Fahrzeug an. Das Adaptive cruise control ist dabei als einfacher proportional regler umgesetzt.

Das autonome Fahrzeug bewegt sich dabei ebenfalls auf eine festen Pfad, dieser führt von rechts in den Kreisverkehr ein und verlässt den Kreisverkehr an der dritten ausfahrt.
die Aufageb des Fahrzeues ist dabei sich sicher in den Kreisfehr einzusortieren und den Kreisverkehr sicher zu verlassen. Dabei Muss das Fahhzeug sowohl aud den Fußgänger, den Radfahrer
und die Andren Fahrzeuge innerhalb des Kreisverkehres achten. Dazu ist das Fahrzeug mit einem virtuellen Velodyne VLP-16 ausgestattet.

\subsection{Simulation Logic}

Um das zuvor beschrieben Scenario durchzuführen musste eine Logik entwicklt werden, diese umfasst nicht nur die Statemachine zum durch fahren des Kreisverkehrs, sondern
auch die Lokalisierung des Fahrzeuges in dem Scenario als auch die Anbinung aller Sensoren an die Objekt Erkennung. Da alle nachfolgenden Berchungen nur wenig resscourcen bentigen
wurde die entsprechende Software in Pyhon entwickelt.

\subsubsection{Sensor Connection}
Die Objekt detection benoötigt als Sensor eingange die Daten des Velodyne VLP-16 und die Daten des Applanix POS-LV. Von den Applanix daten werden jecoch nur
due Aktuelle Postion im WGS84 format benötigt und das aktuelle Heading. Da sich der Applanix Sensor in VREP nicht einfach nachbauen lässt,
wird das Applanix system einfach aus den in Vrep auslesbaren Positions und Rotationsdaten generiert. Da die Position in VREP in Kartesischen Koordinaten
angeben ist, werden dies über die in OpenDAVINCI enhaltene Transformation anhand einer Referenzkoordinate in WGS84 Koordinaten transformiert.

Die Anbindung des Velodyne-VLP gestaltet sich schwieriger, da der Aufabau der Messdaten sich signifikant vom Originalvelodyne unterscheidet.
Zwar werden die Daten ebenfalls in Polarkoordinaten ausgegeben, jedoch werden nur Messwerte ausgegen, wenn diese auf ein Onjekt treffen. Da ein Großteil des Scenarios leerer Raum ist
weisen die Messdaten viele löcher auf, wesshalb diese in eine geeignete Form transformiert werden müsssen.

Dabei ist zu beachten, das das Onjekttracking eine vollständige 360 Gram Messung erwartet, dazu muss zusätzlich gewartet werden, bis alle Nötigen Messdaten vorliegendn.
Der virtualle Velodynle liefet eine Messrate von 10Hz, und ist dabei in 4 Segmente eingeteilt, welche nacheinander ausgelesen werden. Der zur simulation verwendete Zeitschritt
beträgt 50ms, so dass wir 2 Segemente in einem Zeitschritt auslesen können. Solbald alle Daten gesammelt sind, werden diese zu einem geeigneten Messframe zusammengesetzt.

Die glieferten Messdaten liegen als liste von Kugelkoordinaten vor (Radius $r$, Polarwinkel $\theta$, Azimutwinkel $\Phi$). Da die Azimut und Polarwinkel dabei nicht in Aquidistanzen Abständen
vorliegen, jedoch eine sehr viel höhere Auflösung als der originale Sensor haben, werden die Messwerte auf die Originale auflösung von 0.2 bzw. 2 Grad heruntergerundet und die Messerte in ein
Entsprechendes zweidimensionales Array fester größe eingetragen eingetragen. Sodass nicht erfasste bereiche den Wert Null erhalten.

Nach dem alle Erforderlichen Messdaten gesammelt und umgewandelt wurden, werden diese über ein seperat  entwickeltes OpenDAVINCI-Python interface an die Objekterkennung gesendet.


\subsubsection{Mapping and Statemachine}

Laut RASt \cite{rast06} soll ein Kreisverkehr möglichst kreisförmig sein, der einfachheit halber nehmen wir ür die Simulation an, dass dieser ein perfekter kreis ist.
Das heißt, dass Sowohl Straßen, Radwege und fußwege perfekt kreisförmig sind und duch einen Radius beschrieben werden können.








\chapter{Evaluation}



\section{Simuation}
\section{Real Measurements}
\chapter{Conclusions}
Kann in mehrere Unterkapitel gegliedert werden\\\\
Greift Thesen oder Fragestellungen aus der Einleitung wieder auf\\\\
Fasst die Arbeit knapp und prägnant zusammen\\\\
Ordnet die Ergebnisse in Gesamtzusammenhänge ein\\\\
Zieht Schlussfolgerungen aus den erarbeiteten Ergebnissen\\\\
Kann auch eigene Bewertungen oder Meinungen enthalten\\\\
Gibt eine Ausblick auf mögliche Konsequenzen oder notwendige weitere zu lösende Probleme
\chapter{Future Work}

\printbibliography

\end{document}          
