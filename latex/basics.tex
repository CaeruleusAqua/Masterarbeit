\chapter{Related Work}

\section{Roundabouts in Law}
% In Deutschland gibt es kein Gesetzt was den genauen Konstuktion von Kreisverkehren vorschreibt.
% Stattdessen werden die Elemente der Landstraßen und Stadtstraßen in Richtlinien für die Anlage von Landstraßen (RAL) bzw.
% den Richtlinien für die Anlage von Stadtstraßen (RASt) behandelt. Diese Richtlinien sind auch für die
% Wahl einer zweckmäßigen Knotenpunktart bei der Verknüpfung von Straßen maßgebend. Für diese Abreit sind die 
% Richtlinien für die Anlage von Stadtstraßen (RASt) relevant. Die dort behandelten Abwägungsüberlegungen orientieren sich an verkehrlichen Größen, umfeldbezogenen Merkmalen,
% wirtschaftlichen Kriterien und raumordnerischen oder städtebaulichen Vorgaben. Die Richtlinien regeln auch grundlegend die entwurfstechnische und betriebliche Ausbildung von Kreisverkehren.
%
In Germany, there is no law stipulating the exact construction of roundabouts.
Instead, the elements of the rural roads and city streets are dealt with in Directives for the Design of rural roads \cite{ral13}
and the Directives for the Design of Urban Roads \cite{rast06}. These guidelines are also relevant to the choice of a convenient junction type when linking roads.
The considerations discussed there are based on traffic variables, area-related characteristics, economic criteria and spatial planning or urban planning requirements. 
The guidelines also regulate the basic design and operational formation of roundabouts.
The  Directives for the Design of Urban Roads \cite{rast06} are relevant for this dispute. Since the access the RASt ist limited, most of the information is coming from
\cite{man06} whereupon RASt is based on.
\subsection{Elements of a Roundabout}

\begin{figure}[!ht]
%%\begin{center}
\caption{Definition of individual design elements and dimensions of a roundabout \cite{man06}}
\includegraphics[width=0.7\textwidth]{bilder/kreisverkehr.png} %70% der Textbreite
\label{roundabout_parts}
%%\end{center}
\end{figure}


\begin{defi}[roundabout island]
The roundabout island is the constructional area in the middle of the roundabout, which is surrounded by vehicles.
For miniature roundabouts, the roundabout island is crossable.
\end{defi}

\begin{defi}[circular path]
%Die Kreisfahrbahn ist die Fahrbahn, die zum Umfahren der Kreisinsel
%dient. Ein ggf. vorhandener Innenring ist verkehrsrechtlich nicht Be-
%standteil der Kreisfahrbahn (VwV-StVO zu §9a V., Rn. 5).

%%TODO
The circular path is the road that serves to drive the roundabout island. An inner ring, if present, is not part of the circular path (VwV-StVO zu §9a V., Rn. 5).
\end{defi}

\begin{defi}[circular ring with ($B_K$)]
% Die bauliche Breite umfasst die Kreisfahrbahn und einen ggf. gepflasterten
% Innenring. Sie ist abhängig vom Außendurchmesser und der angestrebten
% Verkehrsführung (ein- oder zweistreifig). Die Randstreifenbreite orientiert
% sich an der maßgebenden durchgehenden Fahrbahn.
The structural width includes the circular track and a paved inner ring, if any. It is dependent on the outer diameter and the desired traffic routeing (one or two lanes). 
The edge strip width is oriented on the relevant continuous roadway.
\end{defi}

\begin{defi}[outer diameter ($D$)]
The outer diameter is measured at the outer edge of the circular ring. It is the essential measure for describing the size of the roundabout.
\end{defi}

\begin{defi}[inner diameter ($D_I$)]
The inner diameter is the diameter of the roundabout island.
\end{defi}

\begin{defi}[road divider]
% Der Fahrbahnteiler ist die baulich ausgeführte Insel zwischen Kreisausfahrt
% und -zufahrt einer angeschlossenen Straße. Er dient der Trennung der
% Kreisaus- und -zufahrten, der Führung des Verkehrs sowie den Fußgängern
% und Radfahrern als Überquerungshilfe.
%
The road divider is the structurally designed island between the circular exit
and circular driveway. It serves to separate the circular exit and circular driveway, the management of the traffic, as well as the pedestrians and cyclists as cross-bordering aid.
\end{defi}

\begin{defi}[lane width of the circular driveway ($B_Z$) and circular exit ($B_A$)]
% Die Breite der Kreiszufahrt und Ausfahrt wird am Beginn der Eckausrundung gemessen.
The width of the circular driveway and exit is measured at the beginning of the corner.
\end{defi}

\begin{defi}[Corner rounding radius ($R_Z$ and $R_A$) ]
% Dies ist der Radius der Ausrundung am rechten Fahrbahnrand zwischen 
% der Kreiszufahrt und der Kreisfahrbahn. Bei einem Korbbogen mit einer
% Radienfolge aus drei unterschiedlichen Radien ist RZ der Radius R2 des
% mittleren Bogens. Bei der Ausbildung des Fahrbahnrandes als Schleppkurve ist RZ der kleinste Radius des Fahrbahnrandes.
% 
This is the radius of the rounding at the right edge of the road between the circular driveway and the circular path.
For a elliptical arch with a radius sequence of three different radii, $R_Z$ is the radius $R_2$ of the central arc.
When the road edge is formed as a tractrix, $R_Z$ is the smallest radius of the road edge.
\end{defi}

\subsection{Types of Roundabouts}
There are several types of roundabouts, which are differentiated by the different application criteria and the partly different design principles according to the situation inside and outside built areas.
Furthermore, a division is made as a function of its size.


\subsubsection{Mini Roundabout}

\begin{figure}[!ht]
%\begin{center}
\caption{Mini Roundabout \cite{man06}}
\includegraphics[width=0.5\textwidth]{bilder/mini_roundabout.png} %70% der Textbreite
\label{roundabout_mini}
%\end{center}
\end{figure}

Within built-up areas, smaller outer diameters are possible under certain conditions.
These roundabouts are called mini roundabout. The roundabout island must then be capable of being passed over.
The outer diameter should be at least 13 m, so that the circular island does not become too small.
Larger outer diameters make driving easier. Outer diameters of more than 22m, however, do not offer any transport advantages.
From an outside diameter of about 22 m, therefore, the installation of a small roundabout with 26 m is generally more convenient.
Bypasses are generally not required in the areas where mini roundabout can be used.


\subsubsection{Small Roundabout}

\begin{figure}[!ht]
%\begin{center}
\caption{Small Roundabout \cite{man06}}
\includegraphics[width=0.5\textwidth]{bilder/small_roundabout.png} %70% der Textbreite
\label{roundabout_small}
%\end{center}
\end{figure}

The small roundabout has a single lane circular path and single lane circular driveways and exits. The roundabout island is not passable.
The outer diameter must be at least 26 m. Bypasses can be set up for driving geometric reasons or to increase performance.


\subsubsection{Two-lane Passable Roundabout}

\begin{figure}[!ht]
%\begin{center}
\caption{Two-lane Passable Roundabout \cite{man06}}
\includegraphics[width=0.5\textwidth]{bilder/twolaned_roundabout.png} %70% der Textbreite
\label{roundabout_twolaned}
%\end{center}
\end{figure}

% Reicht die Kapazität des Kleinen Kreisverkehrs nicht aus und kann diese nicht durch die Anlage von Bypässen sicher gestellt werden,
% kann die Kreisfahrbahn eines Kleinen Kreisverkehrs zweistreifig befahrbar ausgebildet werden.
% An einem solchen Kreisverkehr ist die Kreisfahrbahn so breit, dass Pkw im Kreis nebeneinander fahren können.
% Wird eine weitere Erhöhung der Kapazität erforderlich, können einzelne Kreiszufahrten ebenfalls zweistreifig ausgeführt werden,
% wenn Fußgänger und Radfahrer regelmäßig nicht zu berücksichtigen sind. Kreisausfahrten werden aus Sicherheitsgründen immer einstreifig ausgeführt.
% Aus geometrischen Gründen muss der Außendurchmesser bei zweistreifiger Befahrbarkeit mindestens 40 m betragen.
%
If the capacity of the small roundabout is not sufficient and can not be ensured by the installation of bypasses,
the circular path of a small roundabout can be designed to be two-lane driveable.
At such a roundabout, the circular path is so wide that cars can travel side by side in a circle.
If a further increase in the capacity is required, individual circular driveway can also be carried out in two lanes, if pedestrians and cyclists are not to be considered regularly.
For safety reasons, circular exits are always carried out in single lanes.
For geometrical reasons, the outer diameter must be at least 40 m for two-laned accessibility.


\subsubsection{Large Roundabout}


\begin{figure}[!ht]
%\begin{center}
\caption{Large Roundabout \cite{man06}}
\includegraphics[width=0.5\textwidth]{bilder/large_roundabout.png} %70% der Textbreite
\label{roundabout_large}
%\end{center}
\end{figure}

%Große Kreisverkehre mit zwei oder mehreren durch Markierungen gekennzeichnete Fahrstreifen auf der Kreisfahrbahn sollen bei enger
%Abstimmung zwischen Knotenpunktentwurf und Verkehrssteuerung nur mit Lichtsignalanlage betrieben wer den.
Large Roundabouts with two or more lanes marked by markers on the circular path should be operated with a light signaling system only,
if the nodal point design and traffic control are closely coordinated.