\chapter{Introduction}

Das autonome Fahren und die Vernetzung von Fahrzeugen mit Ihrer Umwelt sind zusammen mit der Elektromobilität die meistdiskutierten Themen der Automobilbranche.
Zu Recht: Autonomes Fahren besitzt das Potenzial, im Mobilitätsmarkt völlig neue Strukturen entstehen zu lassen.
\footnote{\url{https://www2.deloitte.com/de/de/pages/consumer-industrial-products/articles/autonomes-fahren-in-deutschland.html} (03/09/2017)}
%%https://www2.deloitte.com/de/de/pages/consumer-industrial-products/articles/autonomes-fahren-in-deutschland.html

\section{Motivation}

Da die Technische Hochschule Chalmers ergänzend zu Volvos “DriveMe” Projekt das Projekt
“CampusShuttle” initiiert, “CampusShuttle” ist ein interdisziplinäres Forschungsprojekt
der Technischen Hochschule Chalmers und der Universität Göteborg.

Das Projekt ist dabei im ReVeRe (Chalmers Research Vehicle Resource) angesiedelt. Die Vision ist dabei ein selbstfahrendes
Auto zwischen den beiden Campus der Technische Hochschule Chalmers.

Dabei soll, im Rahmen des Projekts, das Fahrzeug in verschiedenen Verkehrsszenarien untersucht
werden. Der Fokus liegt dabei besonders auf den Stadtverkehr, das Fahrzeug muss dabei nicht
nur in der Lage sein mit anderen Autos zu interagieren, sondern ebenfalls mit Straßenbahnen,
Bussen, Fahrrädern und Fußgängern sicher agieren. 

\section{Research Question}